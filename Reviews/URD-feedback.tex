% The structure of this document is conform the latest version of the User Requirements Document.

% Title page -------------------------------------------------------------------------------------------------------------------------------
\section{Title page}

\subsection{External reviews}
\subsubsection*{Junior Management}
\begin{longtable}{l|l|p{0.5\textwidth}}
Location/reference & Category & Remark\\
\hline
\hline
\endhead
\hline
\endfoot
title page & Missing & Maybe include student numbers and room numbers for staff (as in other URD's). \\
\end{longtable}

\subsubsection*{Technical Advisor}
\begin{longtable}{l|l|p{0.5\textwidth}}
Location/reference & Category & Remark\\
\hline
\hline
\endhead
\hline
\endfoot
title page & Missing & The advisor is not just an advisor: more explicitely, he is a technical advisor. \\
\end{longtable}

% Abstract ---------------------------------------------------------------------------------------------------------------------------------
\section{Abstract}

\subsection{External reviews}

\subsubsection*{Junior Management}
\begin{longtable}{l|l|p{0.5\textwidth}}
Location/reference & Category & Remark\\
\hline
\hline
\endhead
\hline
\endfoot
Abstract & Missing & State that this document was based on discussions with your client and that it conveys their wishes. \\
\end{longtable}

\newpage

\subsubsection*{Technical Advisor}
\begin{longtable}{l|l|p{0.5\textwidth}}
Location/reference & Category & Remark\\
\hline
\hline
\endhead
\hline
\endfoot
Abstract & structure & Change "how it should function and in what environment it should function" to "how and in what environment it should function". \\
\end{longtable}

% Chapter 1 --------------------------------------------------------------------------------------------------------------------------------
\section{Chapter 1}
\subsection{External reviews}
\subsubsection*{Junior Management}
\begin{longtable}{l|l|p{0.5\textwidth}}
Location/reference & Category & Remark\\
\hline
\hline
\endhead
\hline
\endfoot
1.1 & Structure & You might consider rephrasing the sentence "All of the listed requirements". We just assure that some requirements will be implemented, but not all requirements. \\
1.1 & Other & The sentence "mixing on a mobile device" implies that you are bound to specific hardware. The application should be cross-platform. \\
\end{longtable}

\subsection{Internal reviews}
\subsubsection*{Tessa Belder}

\begin{longtable}{l|l|p{0.5\textwidth}}
Location/reference & Category & Remark\\
\hline
\hline
\endhead
\hline
\endfoot
1.3 & Missing & URD is missing from the list
\end{longtable}

\subsubsection*{Femke Jansen}

\begin{longtable}{l|l|p{0.5\textwidth}}
Location/reference & Category & Remark\\
\hline
\hline
\endhead
\hline
\endfoot
abstract & missing & The abstract is very short and could use some more information; see the template for scmp.pdf as an example. You could mention in the abstract that this document is part of the SEP project and what can be found in this document (a very brief description). \vspace{1em} \\
1.1 & Structure & I would change the beginning to this sentence to "This document", because you already mentioned in the abstract that this document is the URD.\vspace{1em} \\
1.2 & Question & I believe that the client mentioned that the original goal was to find the optimal mix as soon as possible. Should we mention that here as well? \vspace{1em} \\
1.5 & Typo & "remainder" should be "remaining" in "The remainder chapter"\\ 
1.5 & Structure/layout & I would change the layout/text in the list to something like:
* The relation to other systems (2.1)
* The main capabilities (2.2)
* etcetera.
That would make the list more readable, I believe.\\
\end{longtable}

% Chapter 2 --------------------------------------------------------------------------------------------------------------------------------
\section{Chapter 2}
\subsection{External reviews}
\subsubsection*{Junior Management}
\begin{longtable}{l|l|p{0.5\textwidth}}
Location/reference & Category & Remark\\
\hline
\hline
\endhead
\hline
\endfoot
2.1 & Inconsistent & In this paragraph you talk about mobile devices, while it should be cross-platform. \\
2.1 & Other & The sentence "after which output should be shown on the screen" is unclear and ambiguous. \\
2.1 & Other & You talk about sending constraints to the server as a black box, but I think there's too much detail about the black box and how it will work. Just say it is present and you leave the complicated computations to them and you do the visualizing and formatting. \\
2.2.1 & Incorrect & The system should not be able to simulate the flow, as you just said that you leave the computations to the server. You will only visualize the results of these computations. \\
2.2.1 & Other & The sentences "given some constraints. There are a number of constraints to be specified" are redundant, as you already said that there are some constraints in the first sentence. \\
2.2.1 & Other & The sentence "can be specified by tapping on and dragging over the screen" is unclear and ambiguous. What you're actually doing is making an app that can specify these constraints (it should be noted somewhere before additional capabilities). \\
2.2.2 & Missing & From the sentence "When the initial parameters have been set, the computations are offloaded to a server" it is not clear what your capabilities are. Your capability is sending these to the server and retrieving results. \\
2.2.2 & Other & "should be exportable to ... .png or .pdf" is very specific. Maybe there's a better way you didn't think of yet. Here you should probably say: "We need to be able to export it to a standardized animated format (solution space rather than problem space). \\
2.2.2 & Missing & It is in general not clear that we can actually visualise the returned results.\\
2.3 & Other & You use a lot of pretty words and long sentences, but actually you're not really saying what you want to say. For instance, just say that you assume that the server computation does not take too long and thus visualisation can be handled quickly.  That clarifies things much more and the sentences also get shorter. \\
2.4 & Missing & This section is more for different stakeholders. Although in this case, I feel there is actually only one. Still, it might be beneficial to describe WHO your user is (in terms of position, not person) on top of what this person can do. \\
2.4 & Missing & In the sentence "after the application has sent these off to the server, should be able to view the results" it is unclear which results you're talking about. \\
2.5 & Question & You say "intermediate results are sent back to the mobile device for displaying", but is this really the case? Does the server not simply finish a step and then send it back, then calculate the next step? "While solving, return some stuff" is a bit vague. \\
2.5 & Missing & Maybe you can make one of those domain models, even though it will be small. \\
2.6 & Question & "Therefore, we assume this server always answers within a few seconds", but you should also have some basic error handling at least, right? I think it is not a good assumption that the server ALWAYS responds within a few seconds.  \\
2.6 & Missing & What about the assumption that the server in fact gives a correct solution? Or the assumption that the mobile device doesn't lose connection to the internet? \\
\end{longtable}

\subsubsection*{Technical Advisor}
\begin{longtable}{l|l|p{0.5\textwidth}}
Location/reference & Category & Remark\\
\hline
\hline
\endhead
\hline
\endfoot
2.1 & Question & In the sentence "we are to use it as a black box to which we can send constraints and a vector", it is unclear what the vector is. Is this an implementaition detail? If this is the case, it should not be mentioned in this document \\
2.2.1 & Question & What do you mean with the sentence "It is possible to both specify ..."? Perhaps you can insert a few comma's in this sentence or you can re-write it to make it more clear. \\
2.2.2 & Missing & In the sentence "A history of past simulations ... to compare previous runs with the current", I would change "current" to "current run", because now it is unclear what "current" is.\\
2.3 & Inconsistent & "The user interface ... without much hassle" contains both present time and past time writing. Try to stick to writing in one particular time.\\
2.3 & Structure & In general, I see the word "so" quite often, but try to use some other words for "so" as well. \\
2.5 & Question & Where does "parameters described above" refer to? It doesn't appear to be in the previous paragraph in this section, so try to be more clear in which section you described these parameters. \\
2.6 & Typo & In "As a mentioned", "a" should be "we", I believe. \\
2.6 & Missing & In the sentence "Therefore, we assume this server..." I would add the word "that", to obtain the sentence "Therefore, we assume that this server...". \\
\end{longtable}

\newpage

\subsection{Internal reviews}
\subsubsection*{Tessa Belder}
\begin{longtable}{l|l|p{0.5\textwidth}}
Location/reference & Category & Remark\\
\hline
\hline
\endhead
\hline
\endfoot
2.2 & Missing & The sentence "The third parameter to be specified is " is unfinished.\\
\end{longtable}

\subsubsection*{Roel van Happen}
\begin{longtable}{l|l|p{0.5\textwidth}}
Location/reference & Category & Remark\\
\hline
\hline
\endhead
\hline
\endfoot
2.2 & Inconsistent & Third parameter is not exactly a parameter for the algorithm.\\
2.4 & Typo & "The user can then store these results to reference later" $->$ "The user can then store these results for reference later."\\
\end{longtable}

\subsubsection*{Femke Jansen}
\begin{longtable}{l|l|p{0.5\textwidth}}
Location/reference & Category & Remark\\
\hline
\hline
\endhead
\hline
\endfoot
2.2 (description) & Missing & In this paragraph, the "movement of the walls" is mentioned, but we didn't mention the "walls" as a mixing protocol in earlier sections. Perhaps changing it to something like this will clarify it a bit more: ".. given some constraints and initial concentration of the fluids. The constraints are described through a mixing protocol and consists of a sequence of transformations that are applied to the selected geometry. In case of a rectangular geometry, for instance, the sequence consists of 'wall movements': the fluid can be manipulated by moving an upper and lower wall for a specified amount of steps."
\\
2.2 & Structure & I would explain the example as a new sentence, instead of placing it within brackets in the current sentence. In general, using long sentences within brackets can decreas the readability of the text.\\
2.3 & Missing & A third parameter is started to be explained, but the sentence is not finished.
"The third parameter ..." \\
2.4 & Structure/Layout & I would change this to "As mentioned before". Now, it seems like the focus lies on the "being documented" part, whereas you probably only want to say that you mentioned this earlier in this chapter.\\
\end{longtable}

\subsubsection*{Hugo Snel}

\begin{longtable}{l|l|p{0.5\textwidth}}
Location/reference & Category & Remark\\
\hline
\hline
\endhead
\hline
\endfoot
2.1 & Other & "Should be provided to" -$>$ "allows to". We're not provided with any interface\\
2.1 & Inconsistent & "initial details" -$>$ initial concentration distribution. This is consistent with chapter 3.\\
2.2 & Inconsistent & "is the protocol for moving the mixer"-$>$ "is the protocol for moving the geometric component"\\
2.4 & Typo & "results to reference later" -$>$ "results for later reference"\\
2.5 & Structure/Layout & "Apple iPhones and Android phones or tablets" -$>$ "Apple and Android mobile devices"\\
2.5 & Other & "the hard work of computing the matrices" -$>$ "the hard work of computing". The fact that matrixes are used is an implementation detail that the reader should not be aware of in "Environment description".\\
2.5 & Structure/Layout & "will be distributed to" -$>$ "send to". You distribute to $>$1 servers , you send it to 1 server.\\
\end{longtable}

% Chapter 3 --------------------------------------------------------------------------------------------------------------------------------
\section{Chapter 3}
\subsection{External reviews}
\subsubsection*{Junior Management}
\begin{longtable}{l|l|p{0.5\textwidth}}
Location/reference & Category & Remark\\
\hline
\hline
\endhead
\hline
\endfoot
3 & Incorrect & The sentence "any requirements following from further requests will be added here" is not correct: we won't just add any requirements that are requested. Also, after signing the URD, there should not really be any more changes/additions. \\
3.1 & Structure & It might be nice to have more hierarchy for the requirements (as in the review checklist: coherent groups). \\
3.1 & Inconsistent & Formats are named in these constraints, while others are named in previous chapters (.SVG here, .GIF there). Try to make this consistent. \\
CPR03 \& CPR04 & Other & Maybe you can split these requirements, so you can always try to achieve one of them. \\
CPR06 \& CPR17 & Other & Why do you use a list? This seems like a solution space instead of a problem space.\\
CPR09 \& CPR18 & Question & What does "reset" mean here? \\
CPR10 & Other & This requirement is ambiguous: can he save by drawing or can he save a drawing? \\
CNR02 \& CNR03 & Other & Maybe you can split-up the browsers, so you can do them individually. \\
CNR07 \& CNR08 \& CNR09 & Other & We cannot ensure that this is ALWAYS the case. Maybe the "mean/average access time" is a better term. \\
CNR10 & Other & The sentence "should be easily extendable" is ambiguous. Maybe there are more wishes in the Software Engineering sense, such as easily extending mixing patterns, easily extending it for different platforms, easily scaling things (but perhaps with a lower priority). \\
\end{longtable}

\subsubsection*{Technical advisor}
\begin{longtable}{l|l|p{0.5\textwidth}}
Location/reference & Category & Remark\\
\hline
\hline
\endhead
\hline
\endfoot
CPR04 & Question & You mention that the "user can define an initial concentration distribution with black and white", but what are these "black and white" exactly? Do they refer to colors? \\
CPR21 & Typo & "an image of the endresult" should be "end-result".\\
CPR24 & Question &What do you mean in the sentence "an animation of applying the mixing protocol"?\\
CPR29 & Typo & "mixng performance" should of course by "mixing performance", I assume. \\
\end{longtable}

\subsection{Internal reviews}

\subsubsection*{Tessa Belder}
\begin{longtable}{l|l|p{0.5\textwidth}}
Location/reference & Category & Remark\\
\hline
\hline
\endhead
\hline
\endfoot
3.1 \& 3.2 & Missing & Explanation is missing about what exactly capability/constraint requirements are.\\
\end{longtable}

\subsubsection*{Roel van Happen}
\begin{longtable}{l|l|p{0.5\textwidth}}
Location/reference & Category & Remark\\
\hline
\hline
\endhead
\hline
\endfoot
CPR 5 \& 7 & Question & Isn't 7 an extension of 5?\\
CPR 8 & Other & Vague\\
CPR10 & Incorrect & Isn't 6 supposed to be 9?\\
CPR13 & Incorrect & Isn't 9 supposed to be 12?\\
CNR 10 & Inconsistent & It has priority 'should have', while the interface is required for this function has priority 'could have'.\\
\end{longtable}

\subsubsection*{Femke Jansen}
\begin{longtable}{l|l|p{0.5\textwidth}}
Location/reference & Category & Remark\\
\hline
\hline
\endhead
\hline
\endfoot
3.1 & Typo & "off" should be "of" in "and constraints off the application ..."\\
3.1 (CPR 10 \& 13) & Incorrect & CPR10 should refer to CPR09 and CPR13 should refer to CPR12.\\
3.2 & Incorrect & The order/IDs of the constraint requirements are incorrect, as we have duplicate IDs. All IDs must be incremented by one starting from the CNR13 I mentioned.
starting from CNR13 'should have'\\
\end{longtable}

\subsubsection*{Hugo Snel}
\begin{longtable}{l|l|p{0.5\textwidth}}
Location/reference & Category & Remark\\
\hline
\hline
\endhead
\hline
\endfoot
3.1 & Structure/Layout & The priorities should be listed in list-format instead of the current 'newline-format'\\
3.1 & Other & "could have" talks about 'budget' whilst there is no financial compensation. This seems a bit odd.\\
3.1 & Incorrect & "won't have" talks about "this version", whilst the requirements describe only the final deliverable. This seems a bit odd.\\
3.1 \& 3.2 & Structure/Layout & Forall requirements with 'save' --$>$ 'Save and manage'. If you have access to store it, access to delete is kind of implicit (in b4 memory overflow). This would make all 'is able to delete ..." requirements obsolete.\\
3.1 \& 3.2 & Structure/Layout & Sort the table on importance. First all "must have", then "should", then "could".\\
CPR 7 & Typo & should be "..can define a Stepsize" 	or	"..can define a timeperiod."	Not both.\\
CPR 7 & Missing &  Say something about a "fixed period" (otherwise the difference with the dynamische stepsize in 08 isn't clear).\\
CPR 8 & Missing & Mention that stepsize can be adjusted in between steps (aka dynamic stepsize)\\
CNR 12,13,14 & Structure/Layout & replace with "The application is compatible with devices running .."\\
\end{longtable}

% Appendix A -------------------------------------------------------------------------------------------------------------------------------
\section{Appendix A}
\subsection{External reviews}
\subsubsection*{Junior Management}
\begin{longtable}{l|l|p{0.5\textwidth}}
Location/reference & Category & Remark\\
\hline
\hline
\endhead
\hline
\endfoot
Appendix A & Missing & Add alternatives, if necessary. \\
A.2 & Questions & What are the details in the storage of step 6? Will the exports of the results also be deleted? \\
\end{longtable}

\subsubsection*{Technical Advisor}
\begin{longtable}{l|l|p{0.5\textwidth}}
Location/reference & Category & Remark\\
\hline
\hline
\endhead
\hline
\endfoot
Appendix & Structure & Perhaps you can ommit the preconditions and instead include the necessary use cases as an additional step in the use case.\\
Appendix & Missing & Now, it is not directly clear how the use cases are related; perhaps you could add a Use Case Diagram with a "precedes" relation to indicate the preconditions.\\
A.3 & Incorrect & The first step in this use case is that the \applicationname gives a response, but now it seems that the application triggers the use case. This seems a bit strange, because it is a use case for users.\\
A.3 \& A.5 & Question & Is the confirmation message in step 7 from A.3 and step 7 from A.5 the exact same confirmation message? You're mixing general descriptions with specific descriptions, try to stick to one of them.\\
A.7 & Question & Is the priority of this use case correct? It seems that selecting a geometry and mixer is quite important for the application. \\
A.9 & Question & What does the \emph{Load $C_0$} button do? It is not clear what it does from the name of this button. \\
A.10 & Question & Why is the first step numbered as 0? Is it different from the other use cases? \\
A.10 & Structure & I would change "Increments/decrements value in the display..." to "Increments/decrements value and displays the change in the display..." \\
A.11 \& A.12 & Inconsistent & The numbering from the use cases mentioned in step 3 appear to be inconsistent with the actual use cases listed.
\end{longtable}
