% Categories: Question, Typo, Incorrect, Missing, Structure/layout, Inconsistent, Other
% The structure of this document is conform the latest version of the Software Configuration Management Plan.
\chapter{ADD Feedback}
\label{chap:add-feedback}

% Abstract ---------------------------------------------------------------------------------------------------------------------------------
\section{Abstract}
No feedback has been given on this part of the document.
%\subsection{External reviews}
%\feedbackTableStart{Junior Management}{0.0}
%\end{longtable}

%\subsection{Internal reviews}
%\feedbackTableStart{Junior Management}{0.0}
%\end{longtable}

% Chapter 1 --------------------------------------------------------------------------------------------------------------------------------
\section{Chapter 1}
%\subsection{External reviews}
%\feedbackTableStart{Junior Management}{0.0}
%\end{longtable}

\subsection{Internal reviews}
\feedbackTableStart{\femke}{0.0}
1 & Missing & Every chapter with (sub)sections should contain a small introduction before the first (sub)section. \\
1.1 & Typo & In the sentence ``it describes the dependencies on other components'', it should be ``with''  or ``to'' instead of ``on''. \\
1.2 & Typo & In ``The application serves as an education tool'', it should be ``an educational tool''. \\
1.3 & Typo & ``and in perticular for students'' should be ``particular''. \\
1.3 & Typo & The name of the section contains a typo, because it says ``abbrviations'' and it should be ``abbreviations''. \\
1.3.1 & Missing & Now, there is one definition in the list and a todo to add more if necessary. Perhaps you can include definitions for the following items as well: Fortran, mixing step, meta-parameter, GWT and subordinates. \\
1.3.2 & Missing & The abbreviation ``ADD'' is mentioned in this chapter, but it is not included in the list of abbreviations. \\
1.3.2 & Incorrect & 2IP35 is a Software Engineering Project, not a Course. \\
1.5 & Style & Sometimes you are very brief in this section, such as ``Chapter 2 gives a system overview'', whereas sometimes you explicitly say in which sections (``In section 4.2'') something is covered. Perhaps you can include some more information on chapters 2, 3 and 4. \\
\end{longtable}

\feedbackTableStart{\thom{}}{0.0}
1.3.2 & Structure & The abbreviations should be sorted alphabetically, I think. \\
1.3.2 & Missing & The abbreviation ``ADD'' is missing.\\
1.5 & Typo & In the sentence ``...of the system, and an overview...'' there should not be a comma. \\
\end{longtable}

% Chapter 2 --------------------------------------------------------------------------------------------------------------------------------
\section{Chapter 2}
%\subsection{External reviews}
%\feedbackTableStart{Junior Management}{0.0}
%\end{longtable}

\subsection{Internal reviews}
\feedbackTableStart{\thom{}}{0.0}
2 & Structure & The abbreviations URD and SRD can be used here without writing it all out, especially since these abbreviations are named in the list of abbreviations. When you do want to keep the full titles, please emphasise them. \\
\end{longtable}

% Chapter 3 --------------------------------------------------------------------------------------------------------------------------------
\section{Chapter 3}
%\subsection{External reviews}
%\feedbackTableStart{Junior Management}{0.0}
%\end{longtable}

\subsection{Internal reviews}
\feedbackTableStart{\femke}{0.0}
3.1 & Question & Is it really necessary to make a separate section? You're actually only discussing one item in this chapter. \\
3 & Missing & Every chapter with (sub)sections should contain a small introduction before the first (sub)section. That is, if you decide to keep the section for this single item. \\
3.1 & Layout & You provide some sort of list/enumeration here, so perhaps you can use a bullet list. \\
3.1 & Typo & In ``to be done by Fingerpaint applicationare'' a space is missing after ``application''. \\
3.1 & Typo & ``fortran'' should be written with a capital letter. The same holds for the ``fortran'' in the title of this section (if you decide to keep the section). \\
3.1 & Typo & ``Communications'' sounds a bit odd, I think you mean ``Communication'' here. \\
3.1 & Incorrect & The geometry-parameter is not a number, but a String in our current program. \\
3.1 & Incorrect & The matrix/mixer is also a String-parameter. \\
\end{longtable}

% Chapter 4 --------------------------------------------------------------------------------------------------------------------------------
\section{Chapter 4}
%\subsection{External reviews}
%\feedbackTableStart{Junior Management}{0.0}
%\end{longtable}

\subsection{Internal reviews}
\feedbackTableStart{\femke}{0.0}
4.2 & Question & Maybe you can refer specifically to figure 2.1 of the SRD here, so it is more clear that you're talking about the same components here. \\
4.2.1 & Question & What does ``Layout'' and ``Application State'' mean here? They are not mentioned in figure 2.1 of the SRD, so maybe you can explain what they're about. \\
4.2.1 & Question & In the Client-component, you say that the Client Browser is responsible for updating the Application Persistence. Do you mean Client Persistence here? And what is exactly updated in this case? \\
4.2.2 & Typo & A space is missing behind ``Fingerpaint'' in the first sentence. This causes the dot (.) to appear above the first letter of the next sentence. \\
4.2.2 & Question & What does the `depends on'-relation mean? Does it mean that, for instance the Simulator Service communicates with the Fortran Module or vice versa? Perhaps you can explain some more about this relation, so the ``direction'' of the relation is also clear. Explaining the bidirectional arrows might also help for understanding the figure better. \\
Figure 4.1 & Incorrect & Figure 2.1 in the SRD has bidirectional arrows for all communication lines, for example the communication line between the Simulator Service and the Fortran Module. I think there should be bidirectional arrows in figure 4.1 as well for all these cases. \\
Figure 4.1 & Incorrect & In figure 4.1 you say that the Application Service depends on the Simulator Service, but that seems quite strange, as the Client Browser is the one who communicates to the Simulator Service (see figure 2.1 SRD).  \\
Figure 4.1 & Question & Why does the ``Update Application Persistence'' depend on the HTTP server? If the Application Persistence is actually the Client Persistence, is there any need to communicate to the HTTP server at all? If this is not the case, could you please explain the purpose of the Application Persistence better? \\
\end{longtable}

\feedbackTableStart{\thom{}}{0.0}
4.1 & Missing & When mentioning the use of GWT, maybe refer to the SCMP (chapter 6)? Maybe not, though, because the SCMP is a project document, whereas the ADD is a product document. \\
4.2.2 & Typo & In the first sentence, I think there should be a \texttt{\{\}} behind the \LaTeX{} command for the project name, because there is something weird happening now. \\
\end{longtable}

% Chapter 5 --------------------------------------------------------------------------------------------------------------------------------
\section{Chapter 5}
%\subsection{External reviews}
%\feedbackTableStart{Junior Management}{0.0}
%\end{longtable}

\subsection{Internal reviews}
\feedbackTableStart{\thom{}}{0.0}
5.2.4 & Layout & Under ``Purpose'', I think the layout of the block with SRQs is not correct. It starts with an indentation now, but that should not be the case. Use \texttt{\textbackslash{}fpstartparagraph}. \\
\end{longtable}

% Chapter 6 --------------------------------------------------------------------------------------------------------------------------------
\section{Chapter 6}
No feedback has been given on this part of the document.
%\subsection{External reviews}
%\feedbackTableStart{Junior Management}{0.0}
%\end{longtable}

% Chapter 7 --------------------------------------------------------------------------------------------------------------------------------
\section{Chapter 7}
No feedback has been given on this part of the document.
%\subsection{External reviews}
%\feedbackTableStart{Junior Management}{0.0}
%\end{longtable}

%\subsection{Internal reviews}
