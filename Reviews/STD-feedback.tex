\chapter{STD Feedback}
\label{chap:std-feedback}

% Please only include chapters for which feedback is given!

% Chapter 1 --------------------------------------------------------------------------------------------------------------------------------
\section{Chapter 1}
\subsection{Internal reviews}
\feedbackTableStart{\hugo}{0.0}
abstract &	Inconsistent 	& customer $\rightarrow$ client\\
1.1 &		Inconsistent	& customer $\rightarrow$ client\\
1.3 & 		Inconsistent	& customer $\rightarrow$ client. Basically this comment holds throughout the document. In all other documents we call him "the client"\\
1.3 & 		Other		& "Java Development Kit: The software used to build Java applications" Either 'the software WE used' or 'software used to...'\\
\end{longtable}

% Chapter 2 --------------------------------------------------------------------------------------------------------------------------------
\section{Chapter 2}
\subsection{Internal reviews}
\feedbackTableStart{\hugo}{0.0}
2 & 		Other &		Broken references.\\
\end{longtable}

% Chapter 3 --------------------------------------------------------------------------------------------------------------------------------
\section{Chapter 3}
\subsection{Internal reviews}
\feedbackTableStart{\hugo}{0.0}
3 & 		Other &		Broken references.\\
\end{longtable}

% Chapter 4 --------------------------------------------------------------------------------------------------------------------------------
\section{Chapter 4}
\subsection{Internal reviews}
\feedbackTableStart{\hugo}{0.0}
4 & 		Question & 	We talk about the "project-docs directory" as if we're going to hand over the github repo. Same for 'archive' and 'master library'. At the moment, it is not sure if we're going to hand over repos. Therefore I wouldn't talk about it in this fashion.\\
\end{longtable}

% Chapter 5 --------------------------------------------------------------------------------------------------------------------------------
\section{Chapter 5}
\subsection{Internal reviews}
\feedbackTableStart{\tessa}{0.0}
5 & Typo & Spaces are missing between ATP, URD and their references. \\
\end{longtable}

\feedbackTableStart{\roel}{0.0}
5 & Typo & Broken references to the URD and ATP. \\
\end{longtable}

\feedbackTableStart{\benjamin{}}{0.0}
5 & Other & The commas in the long list of CPRs are bold, when they should not be. Also \emph{and} is in boldface. \\
5 & Other & Have we actually tested non-iOS operating systems? \\
5 & Other & \emph{Another conclusion that came from \ldots}: \emph{came from} is unnecessary. \\
\end{longtable}

\feedbackTableStart{\hugo}{0.0}
5 &		Missing &	I would add "(all versions)" to the description of IE and Safari not working.\\
\end{longtable}

% Chapter 6 --------------------------------------------------------------------------------------------------------------------------------
\section{Chapter 6}
\subsection{Internal reviews}
\feedbackTableStart{\tessa}{0.0}
6 & Other & Maybe add something here like: `If a solution is known for these bugs, but not yet implemented, this will also be mentioned.' And then remove the `Solution' section everywhere where the solution is `The issues should disappear'. \\
6.5 & Typo & The solution should start with a capital letter. \\
\end{longtable}

\feedbackTableStart{\benjamin{}}{0.0}
6 & Other & Transfer phase is mentioned twice. \\
6.1 & Other & \emph{When a popup pops up}......... Yea. \\
6.1 & Other & \emph{When the popup pops up}......... Same as above, but at least be consistent with \emph{a} and \emph{the}. \\
6.2, 6.5, 6.6 & Other & This is the desired effect, not the solution. \\
6.4 & Question & Is this still true? I thought we had already fixed this some time ago. \\
6.7 & Other & Change \emph{the to be downloaded file} to \emph{the file to be downloaded}. \\
\end{longtable}

\feedbackTableStart{\hugo}{0.0}
6 & 		Style &		"During transfer phase"... "at the transfer phase". I would remove one of the two. Also the ":" at the end of the sentence is a bit out of place when there is a section rather than a list on the next line.\\
6.3 & 		Other & 	Say something like "to prevent accidental deletion". Otherwise it's like 'its missing' and 'it should be added' without any reason being given for the change.\\
6.5 &		Other & 	We can be a bit more precise here. Say something like "Get the gwt-visualization.jar library to work for IE"\\
6.6 & 		Other & 	I feel like we can be more precise than 'it should not crash' here. Think in terms of solutions instead of 'it should stop exhibiting the unwanted behavior'. (i.e. "It should resize properly", "It shouldn't be resizable to smaller sizes", "scroll bars should appear", etc).\\
x &         Missing &   We're missing the issue of the fortran code containing a memory leak. It's beyond the scope of our project, and we're not going to fix it, but it's definitely an issue of the Fingerpaint application.
\end{longtable}

% Chapter 7 --------------------------------------------------------------------------------------------------------------------------------
\section{Chapter 7}
\subsection{Internal reviews}
\feedbackTableStart{\benjamin{}}{0.0}
7 & Other & Show loading: busy is superfluous here, the present continuous tense already implies that the activity is being executed at that time. \\
7 & Other & Logarithmic scale: change y-axis to $y$-axis. \\
7 & Other & Gray entries: from reading this document, it is unclear what is meant by \emph{gray} and \emph{selected} entries. \\
7 & Typo & Gray entries: Gray is spelt \emph{grey}. Gray is an SI unit. \\
\end{longtable}

\feedbackTableStart{\hugo}{0.0}
7 & 		Other & 	"In the view multiple performance option, the gray entries in the loading menu are confused
with selected entries. They should be more distinguishable." Selected items should be distinguishable. At the moment grey entries are so distinguishable that they being are confused with selectable items. Hence, they should be made LESS distinguishable (from other non-selected cellList items).\\
\end{longtable}

\section{Chapter 8}
\subsection{Internal reviews}
\feedbackTableStart{\benjamin{}}{0.0}
8 & Other & Gray entries: from reading this document, it is unclear what is meant by \emph{gray} and \emph{selected} entries. \\
8 & Typo & Gray entries: Gray is spelt \emph{grey}. Gray is an SI unit. \\
\end{longtable}
