\chapter{ATP Feedback}

\section{Abstract}
\subsection{Internal reviews}
\feedbackTableStart{\benjamin}{0.0}
General & Typo & Please enable spell checking for your \LaTeX{} editor. \\
Abstract & Question & \emph{\ldots a report on the results needs to be written.} Does this mean that a report should be written after the client has run the tests? Is this report part of this document? I can get the answers from reading the rest of the document, but on its own, the wording is a bit unclear. \\
Abstract & Structure & The last sentence with the ESA standards is unexpected and directly follows unrelated text. I'd move this sentence to the start of the abstract and start a new paragraph for the remaining text. \\
\end{longtable}

\section{Chapter 1}
\subsection{Internal reviews}
\feedbackTableStart{\benjamin}{0.0}
1.2 & Other & The last sentence implies that the user manual is written because the server setup needs to run the tests on its own. \\
1.2 & Other & In the last sentence \emph{Appendix A} is not a clickable reference. \\
1.3.1 & Other & For Safari, you might want to make it clear that this is the version used for desktop systems, to indicate that iOS Safari is not the same as Safari. \\
1.3.2 & Typo & The list contains \emph{At} which should probably be \emph{AT}. \\
\end{longtable}

\section{Chapter 2}
\subsection{Internal reviews}
\feedbackTableStart{\benjamin}{0.0}
2 & Other & The wording of this introduction is strange and contains too much passive forms. \\
2.* & Other & This chapter contains a lot of incorrect citations, e.g. \emph{URD\textbackslash ref\{urd\}} instead of \emph{URD\textbackslash cite\{urd\}}. \\
2.1 & Other & Why do the first two sentences need to be two sentences? These tests are only to test whether the application fulfils the URD requirements. The first sentence need not be there if the second is reworded slightly. \\
2.6 & Other & Broken reference in the last sentence. \\
\end{longtable}

\section{Chapter 3}
\subsection{Internal reviews}
\feedbackTableStart{\benjamin}{0.0}
3 & Other & \emph{Acception} should be something like \emph{exception}, but the sentence should be reworded to fix this. Also, try to avoid \emph{ones}. \\
3.* & Other & Personally I would use \emph{the user} instead of \emph{a user}. \\
3.* & Typo & Various lines do not end with ``.''. \\
3.* & Other & Throughout the chapter, both computer and mobile input actions are specified, which reads awkwardly. I would discard the computer actions, as we have always said that the application is only really supported on mobile devices. If both types of actions should be kept, I would just put a notice at the start of the chapter. \\
3.* & Structure & It might be an idea to create a table with on the left the input and on the right the output. Currently it is slightly annoying to scroll up and down every time to validate whether the correct output follows from input. \\
3.1 & Other & Add a \emph{\textbf{Test items:}} for consistency to the table. \\
3.1 & Missing & After step 2, not only does the canvas appear, but the menu (which is crucial to the application) does too. \\
3.1 & Other & Step 7: \emph{Click/Tap on somewhere in the drawing area} has a superfluous \emph{on}. \\
3.1 & Incorrect & Step 12, 14: There is no \emph{black} or \emph{white} in the menu bar, only a picture with a black and white rounded square. The text makes it appear that there are buttons with these texts. \\
3.* & Other & Occasionally, the \emph{Rest Dist} button is referenced, this should be \emph{Reset Distribution}. \\
3.2 & Missing & This test does more than it says in the test items: it also tests whether all drawing tools work etcetera. \\
3.* & Other & \emph{\ldots there is shown \ldots} sounds \textbf{really} strange. \\
3.3 & Missing & This test also does more than its test items suggest, such as loading and overwriting. \\
3.4 & Missing & At this point, a smiley is drawn on the canvas, but the test assumes the canvas is empty. Somewhere the distribution should be reset. \\
3.* & Incorrect & There is no \emph{\#steps label}, this was changed to \emph{Number of steps} or similar. I think it would suffice to say \emph{the spinner for the number of steps}. \\
3.5 & Typo & In \emph{Test items}, it should say \emph{\ldots define and execute \textbf{a} mixing protocol}. \\
3.* & Incorrect & There is no top button any more for number spinners. \\
3.* & Incorrect & There is no \emph{Define Protocol} check box. \\
3.11 & Question & What is the added value of loading a second initial distribution? \\

\end{longtable}

\section{Chapter 4}
\subsection{Internal reviews}
\feedbackTableStart{\benjamin}{0.0}
4 & Typo & \emph{For the tests to succeed, It is important \ldots}: \emph{It} should not be capitalised. \\
4 & Other & \emph{\ldots unless the test states otherwise. Otherwise some steps \ldots} The second sentence is only a fragment, and the word \emph{otherwise} occurs twice in succession. \\
\end{longtable}

\section{Chapter 5}

\section{Chapter 6}
\subsection{Internal reviews}
\feedbackTableStart{\benjamin}{0.0}
- & Other & The right and left sides of the header of the page overlap. \\
\end{longtable}