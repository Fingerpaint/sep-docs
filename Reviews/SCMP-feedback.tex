\chapter{SCMP Feedback}
\label{chap:scmp-feedback}

% Abstract ---------------------------------------------------------------------------------------------------------------------------------
\section{Abstract}
\subsection{Internal reviews}
\feedbackTableStart{\femke}{0.0}
General Remark & Other & There was already a LaTeX template available for SCMP; perhaps you can check whether you conformed to the template. Another possibility is to merge the pieces you have with the text that is provided in the template. \\
\midrule
Abstract & Missing & The reference to the ESA document doesn't work/is missing. As a consequence, all references mentioned in the document are broken/missing. \\
\end{longtable}

% Chapter 1 --------------------------------------------------------------------------------------------------------------------------------
\section{Chapter 1}
\subsection{Internal reviews}
\feedbackTableStart{\benjamin}{0.0}
1 & Other & The introduction is in future tense, while it should be present tense.\\
1 & Other & The introduction is strangely constructed, and the second sentence should be a subsentence of the first.\\
1.1 & Other & Second sentence contains future tense again.\\
1.4 & Missing & References are empty, and in turn all references show as ``[?]''.\\
\end{longtable}

\feedbackTableStart{\femke}{0.0}
1.3	 & Incorrect & 2IP35 is the Software Engineering Project, not a Course. \\
1.3	 & Other & You don't have to translate BCF: its English equivalent meaning is namely ``Bureau for Computer Facilities". \\
1.4	 & Missing	 & The list of references is empty.
\end{longtable}

% Chapter 2 --------------------------------------------------------------------------------------------------------------------------------
\section{Chapter 2}
\subsection{External reviews}
\feedbackTableStart{Junior Management}{0.0}
2.1 & Question & `Other group members should always assist' -- Unclear. What is meant by `should always assist'? \\
2.4 & Missing & For SCMP and SPMP the references are missing. \\
2.5 & Missing & `should use the fingerpaint.cls document' -- A reference to this document should be added as well. \\
2.5 & Structure/Layout & `This to ensure a consistent...' -- Grammar mistake. Maybe something like ``This [rule is meant] to ensure [...]''?\\
\end{longtable}

\subsection{Internal reviews}
\feedbackTableStart{\tessa}{0.0}
2.5 & Typo & ``So, a standard ''main`` .tex-file should like as is shown'' -- probably should be something like ``should look like as is shown''.\\
\end{longtable}

\feedbackTableStart{\benjamin}{0.0}
2 & Other & First sentence of \emph{Management} is passive, but it could easily be changed to active form.\\
2.2 & Missing & Master and archive libraries are introduced, but they are never explained. Only in Section 3.1 is it mentioned that these will be described in Chapter 4.\\
2.2, 2.3, 2.5, 3.1, 3.2 & Typo & ``chapter'', ``section'' etc. should be capitalised.\\
2.5 & Typo &  It seems a word is missing in \emph{So, a standard ``main'' .tex-file should like as is shown in figure 2.1}.\\
\end{longtable}

\feedbackTableStart{\femke}{0.0}
2.1 & Missing	 & 	Here, you refer to the SPMP, but there is no reference to this external document. \\
2.3	 & 	Missing	 & Again, you mention the SPMP, but there is no reference. \\
2.5	 & 	Question & Is it an idea to make a small figure that summarizes what you say in this paragraph, so people can immediately say what the required structure of the documents is? \\
2.5	 & Missing & The template describes that all documents must also adhere to the requirements in SQAP and SVVP. \\
\end{longtable}

% Chapter 3 --------------------------------------------------------------------------------------------------------------------------------
\section{Chapter 3}
\subsection{External reviews}
\feedbackTableStart{Junior Management}{0.0}
3.1 & Other & `This identifier is title abbreviation-version' -- This was confusing when I read it the first time, especially with the space (it looks like ``title'' is one term and ``abbrevation-version'' is another term). I would write ``[title abbreviation]-[version]'' or something similar maybe. Although that is more a personal opinion. This was also done this way in 6.2.2. by the way. \\
3.1 & Other & `Basically, the version number will not change after that, but it is theoretically possible that after that, some more changes are required and versions 1.x are created.' -- This is a little unclear. First it states that ``[...] the version number will not change [...]'' followed by ``[...] but it is possible that [...] changes are required[...] and versions 1.x are created''. So the version might change apparently. Also, try to avoid using ``after that'' too many times in a row, it can get confusing what ``that'' is. \\
3.1 - Second Paragraph & Structure/Layout & This does not really belong in the naming conventions anymore.\\
3.2 & Missing & `The ESA standard prescribes' -- A reference should be added here. \\
\end{longtable}

\subsection{Internal reviews}
\feedbackTableStart{\benjamin}{0.0}
3.1 & Other & Future tense in first sentence of first and second paragraph.\\
3.2 & Typo & \emph{Baseslines} in second sentence.\\
3.2 & Other & Future tense in third sentence.\\
3.2 & Typo & Last sentence of first paragraph: \emph{rebuild} should be \emph{rebuilt}.\\
\end{longtable}

\feedbackTableStart{\femke}{0.0}
3.1 & Style & The writing style of this section is a bit informal, in my opinion, especially the sentence ``note how exception this situation sounds". It is a formal document, so a formal writing style is appropriate here. \\
3.1 & Typo & ``Et cetera`` should be ``etcetera'', I think. \\
\end{longtable}

% Chapter 4 --------------------------------------------------------------------------------------------------------------------------------
\section{Chapter 4}
\subsection{External reviews}
\feedbackTableStart{Junior Management}{0.0}
4.1.1 & Other & `and not to the SEP that led' -- Unclear. Is a CI not automatically related to the Software Engineering Project (SEP) if it is related to the application you are creating? Maybe saying ``[...] and not exclusively to the SEP documentation [...]'' is less confusing? \\
4.1.3 & Structure/Layout & The last line of the folder structure under the bullet `Documentation' is displayed on the next page. This is ugly formatting. \\
4.3.2 & Question & `Note that in this procedure, the version number of the CI is bumped with 1.0.' -- I am not sure, but shouldn't it say ``bumped to''?\\
\end{longtable}

\subsection{Internal reviews}
\feedbackTableStart{\tessa}{0.0}
4.1.2 & Missing & ``on the website of the Fingerpaint application.'' -- A reference to this site would be nice.\\
4.2 & Typo & ``and how the libraries can accessed through it'' -- should be ``can be accessed through it''.\\
4.3 & Typo & ``who can change what in the various libraries'' -- should be ``what is in the various libraries''.\\
4.3.2. & Question & It says files in the master library can only be changed by the CM, but you have to contact the QM if you want to change something. Why QM and not CM? Or is this just a typo? If QM is correct, it has to be defined somewhere (it's not in the definitions list). I suppose it means quality manager, but i'm not sure.\\
\end{longtable}

\feedbackTableStart{\benjamin}{0.0}
4, 4.1.1 & Other & Future tense.\\
4.1.1 & Other & It says that documents that are stored are stored in \emph{The development library will store all versions of a CI that are stored in it}. I get what is meant, but it does sound strange.\\
4.1.1 & Other & \emph{made undone} should be \emph{undone}.\\
4.1.1 & Other & \emph{(Git is discussed in more detail in chapter 6)} can be a main sentence instead of a subsentence.\\
4.1.1 & Other & \emph{all CIs are stored here that are related to the Fingerpaint application and not to the SEP that led to the creation of that application} should be \emph{all CIs that are related to the Fingerpaint application and not to the SEP that led to the creation of that application are stored here}.
\end{longtable}

\feedbackTableStart{\femke}{0.0}
4 & Style & ``Since we are at it`` sounds a bit informal to me. Perhaps you can change it to, ''Moreover, we will describe ..." \\
4.1 & Style	 & In the sentence ``We call a place where CIs are stored a library", I would place library between quotes, as you introduce it as a new term here. \\
4.1.1 & Style & In project-docs, in the sentence ``That is, all CIs that ... are stored here`` I would place an additional comma somewhere, to make it more readable. It is now unclear to me what the ``leading to the creation of that application''-part refers to. \\
4.1.1 & Style & In sep-docs, I would start the second sentence with ``This`` instead of ``That'': often ``that`` is used in the same sentence, whereas ``this'' is more appropriate in a new sentence, I think. \\
4.1.3 & Question & The purpose of the sentence ``Of course, the difference is clearly stated on the website" is not clear to me. Do we have to mention explicitely that the difference is explained on the website? Now, it appears that the focus lies on the fact that the difference is stored on the website, whereas you actually just want to say that there is a difference, I think. \\
4.1.3 & Missing & The template mentions again SQAP and SVVP, but there are not mentioned in this section. \\
4.3.2 & Missing	 & You mention the SVVP document, but there is no reference to it. \\
4.3.3 & Missing & The template mentions SVVP, but there is no reference to that document in this subsection. \\
4.3.3 & Style & The sentence ``Note that files ... been approved externally" is a bit long, perhaps you can split it in multiple sentences? \\
\end{longtable}

% Chapter 5 --------------------------------------------------------------------------------------------------------------------------------
\section{Chapter 5}
\subsection{Internal reviews}
\feedbackTableStart{\benjamin}{0.0}
5 & Other & Future tense in first sentence.\\
\end{longtable}

\feedbackTableStart{\femke}{0.0}
5 & Question & In the template, it is mentioned that changes in the development library will not be recorded. However, in your table an entry for ``current" is created? \\
\end{longtable}

% Chapter 6 --------------------------------------------------------------------------------------------------------------------------------
\section{Chapter 6}
\subsection{Internal reviews}
\feedbackTableStart{\tessa}{0.0}
6.1.1. & Other & ``to work efficiently in parallel on the same file, even to some extent on the same file'' -- Don't understand what is meant here exactly, but probably not two times ``on the same file''. \\
6.1.2 & Typo & ``GitHub is a (commercial) servers'' -- should be ``server'' (or maybe ``set of servers''?). \\
6.1.3 & Inconsistent & In this section ``the GWT'' is mentioned a few times, and also ``GWT''. I don't think this is very consistent, since ``the GWT'' can also be used at all places where ``GWT'' is used. (except for GWT-enabled). \\
\end{longtable}

\feedbackTableStart{\benjamin}{0.0}
6.1 & Incorrect & Saying that \emph{we will be able to do anything the client wants} is a very strong statement.\\
6.1 & Other & Second paragraph: lots of future tense.\\
6.1.1 & Typo & Superfluous comma in \emph{Git is a distributed, lightweight, version control system}.\\
6.1.1 & Typo & In \emph{GitHub is a (commercial) servers}, servers should probably be service.\\
6.1.3 & Other & In my opinion, writing JavaScript instead of JS looks better.\\
6.1.3 & Missing & The five most widely used browsers are mentioned, but not named.\\
6.1.4 & Other & Like it is written now, \emph{It is simply impossible to test both browsers on \ldots} implies that there are only two browsers.\\
6.1.4 & Other & \emph{It is multilingual, including Java, which is our choice} is a strange fragmented sentence.\\
6.2.1 & Other & The colon in \emph{\ldots so the conventions are simple: a developer does not want to do something complex a lot} implies that ``the conventions are simple, namely a developer \ldots''.\\
\end{longtable}

\feedbackTableStart{\femke}{0.0}
6.1.3 & Question & Is it really the case that GWT provides only two things? Or does it provide many things and are we only going to use those specific two things? \\
6.1.3 & Typo & In the sentence ``Finally, the GWT includes a plugin..``, ``plugin'' should be ``plug-in". \\
6.1.4  & Other & You have two sentences that start with ``Of course``. This reads a bit odd and I think you actually mean to say ``However'' in the second sentence. \\
6.1.4 & Other & It seems a bit strange to start a sentence with ``Still then``, perhaps you can suffice with ``Still''. \\
6.1.6 & Style & The style of the sentence ``if the server .., which we have done`` is again a bit informal. Perhaps you can simply remove the ``which we have done'' part, or you can re-phrase it in a different way. \\
6.1.7 & Question & I understand that we use LaTeX because of the good-looking documents, but is that really a key motivation to use LaTeX? \\
6.2.1 & Style & Again, beware of the informal style: it is a formal document, after all.
\end{longtable}

% Chapter 7 --------------------------------------------------------------------------------------------------------------------------------
\section{Chapter 7}
\subsection{External reviews}
\feedbackTableStart{Junior Management}{0.0}
7 & Other & `In general, when we consider using software from a supplier we trust, we just use the software.' -- I would be careful about stating that this way. You should never blindly trust and use software products. i had occasions where i used well known libraries in JAVA for example, had an error in my code and tried to track it down in my code only, just to find out after days and days that the error was in the library itself. \\
\end{longtable}

\subsection{Internal reviews}
\feedbackTableStart{\femke}{0.0}
7 & Style & The introduction text is slightly informal, try to stick to a more formal writing style. \\
\end{longtable}
