\chapter{Tools, Techniques and Methods}
\label{chap:tools}
In this chapter, the various tools are discussed that are used in the \projectname{} project.

The \applicationname{} is a web application that runs locally in the browser. The client has indicated it should be implemented in HTML5. At the moment, HTML5 is a sort of ``umbrella definition'' for a lot of technologies. In practice, what we will do is implement the application in JavaScript (JS), making heavy use of the canvas element and everything it supports. Note that currently, the canvas element supports quite a lot of features: simple things like drawing rectangles, but also drawing curves and more complex shapes (polygons) is supported. Even creating effects like blur or inverting colors belongs to the possibilities. Finally, 3D-drawing is supported. We will not make use of the latter, but the idea is clear: with a canvas, we will be able to do anything the client wants.

To make developing the software easier, we will use a couple of frameworks. This will enable us to document the code more easily and more clearly (including generating documentation from the comments in the code). Additionally, it will enable automated testing. This will be explained below.

\section{Git}
All code and documentation is stored in Git repositories. Git is a distributed, lightweight, version control system. It enables all team members to work efficiently in parallel on the same file, even to some extent on the same file. Apart from enabling working in parallel, it also provides an implicit backup system, as every team member has a local copy of a repository.

\section{GitHub}
Git is a distributed system in which a central server is needed. This is the ``main'' repository, to which a number of project members can connect (this is called \emph{cloning}, every project member has a \emph{clone} of the ``main'' repository on his/her machine). GitHub is a (commercial) servers that can host the ``main'' repository for us. As long as the repository is public, the service is free to use. The client has expressed that the project can be public, so we can use this free service from GitHub. Apart from providing ``simple'' hosting, GitHub also comes with a nice-looking web interface with which anyone can browse the repository. Also, there is a Wiki and bug-tracking system. Finally, GitHub provides a mechanism for hosting a website by means of creating a repository with a special name. We use this to build a website on which we can present some progress information to the client and maybe documentation or other project-related things.

\section{Google Web Toolkit (GWT)}
The GWT\footnote{The homepage of GWT can be found at \url{https://developers.google.com/web-toolkit/}.} is a toolkit that provides two things:
\begin{itemize}
	\item possibility to develop (complex) browser-based applications productively;
	\item optimised JS generated by GWT provides the user with a high-performance end product.
\end{itemize}
A great advantage of the GWT to us is that it enables developers to create web applications without requiring the developer to be an expert in browser quirks or JS. A developer simply needs to know Java (which is well-known and easy to learn) to understand the code of the application. The heart of GWT, namely, is a compiler that can convert (GWT-enabled) Java code into JS. GWT even generates different versions of the JS for the five most widely used browsers, optimising performance and preventing browser quirks from having effect on the application.

Finally, the GWT includes a plugin for Eclipse and an extension for Chrome to profile the application, which makes developing again more productive. When using the plugin, it is possible to deploy the web application on a local server, keeping the developing cycle very short. If a new feature is implemented, a ``production version'' of the web application can be uploaded to a server.

\section{Selenium}
From the start of the project, we want to be able to test the software automatically. As we are developing a web application here, an important part is user experience. Of course, one of the most important facets of the user experience is the visual facet: what does it look like? Of course, it is very labour intensive to test this manually: that would require a tester to open the application in a number of browsers and perform the same sequence of actions in every browser. Still then, probably just a subset of all available browsers will be tested. It is simply impossible to test both browsers on desktop Windows, Mac and Linux machines and on mobile iOS and Android machines - which are not even \emph{all} platforms available on the market.

Selenium\footnote{The homepage of Selenium can be found at \url{http://docs.seleniumhq.org/}.} is a library that can aid here. It is multilingual, including Java, which is our choice. Selenium provides a means of creating and executing tests in a lot of browsers, including mobile browsers for iPhone and Android. A test in this context is literally a sequence of actions \emph{as the user could perform them}. It is possible to click elements, type text, submit forms, drag-and-drop, simulate clicks, et cetera.

The code base of Selenium also includes a server. This server can be used to run actual tests on a remote machine, while the program that is issuing the tests runs on a local machine. Commands/actions are sent to the server through a tunnel, the server executes those actions in a browser of choice and returns the results to the local machine, when asked for. This functionality is used by us to be able to run tests from any machine on a number of different platforms.