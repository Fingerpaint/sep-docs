\chapter{Management}
\label{chap:management}
In this chapter it is specified which project members are involved in configuration management. Also, the responsibility a team member has in a function involved with configuration management is explained. Then, some general responsibilities are explained. Finally, a template for the creation of documents is given and some conventions are set about document creation, to ensure a consistent layout.

\section{Organisation}
The team members involved in configuration management are the configuration manager (CM) and vice CM. The project members that have volunteered to fulfil these roles are named in the SPMP \cite{spmp}.

Other group members should always assist the CM and vice CM.

\section{Responsibilities}
The CM and vice CM are responsible for copying documents to the master and archive library at the right moments, as mentioned in chapter \ref{chap:configurationcontrol}. They are in general responsible for the contents of the master and archive library. Another task for them is creating and updating document templates, although this task can be delegated.

The CM is primarily responsible for configuration management, although he or she can delegate tasks to the vice CM, in which case the vice CM is responsible. Whenever the CM is (temporarily) not available, the vice CM should take over the tasks of the CM, including the responsibility for these tasks.

Finally, all project members are responsible for the documents they work on. This means that they update the document status sheet and make sure the latest version of the document they work(ed) on is available in the development repository (refer to chapter \ref{chap:configurationcontrol}, section \ref{subsec:configcontrol-library-dev}). When multiple group members work on the same document, they share the responsibility and additionally are responsible for the combined consistency of the document. Also, they should make sure that the repository remains in a ``workable'' state. That is, they should solve possible merge conflicts together.

\section{Interface Management}
\label{sec:interfacemanagement}
The \projectname{} application will be developed using an external virtual server provided by the BCF. In case of failure of this server, the CM will contact BCF and let them resolve the issue. BCF is in this case supposed to have expert knowledge and have the means to resolve issues.

In general, the CM can help other project members when they have trouble with some software that is used (refer to chapter \ref{chap:tools}). However, the CM may delegate this task to other group members who have more expertise on the subject.

\section{SCMP Implementation}
In this project, we will have only one SCMP document, contrary to what is described in the ESA standard \cite{esa}. Thus, this document will not contain a planning for every phase of the project. Instead, refer to the SPMP for the planning of the phases.

\section{Applicable Procedures}
Every non-code document has to be created using \LaTeX{} and should use the \texttt{fingerpaint.cls} document class by declaring the following in the beginning of the document (with other values for the options of course):
\begin{center}\begin{minipage}{0.5\textwidth}\begin{verbatim}
\documentclass[%
    pathtobase=..,%
    titlefull={Full Document Title},%
    titleabbr=FDT,%
    version=0.1]{fingerpaint}
\end{verbatim}\end{minipage}\end{center}
Every document should start with a \verb|\maketitle{}| call, followed by an abstract and then the \verb|\tableofcontents|. This to ensure a consistent style among all documents.

Each document should have a ``main'' \texttt{.tex}-file that \verb|\input|s separate \texttt{.tex}-files that each contain a chapter. This enables project members to work on the same document more efficiently, as working on different chapters will not cause any merge conflicts.

So, a standard ``main'' \texttt{.tex}-file should like as is shown in figure \ref{fig:exampledocument}.

\begin{figure}
	\begin{center}
		\begin{minipage}{0.8\textwidth}
			\begin{verbatim}
				\documentclass[%
				    pathtobase=..,%
				    titlefull={Full Document Title},%
				    titleabbr=FDT,%
				    version=0.1]{fingerpaint}

				\begin{document}

				  \maketitle{}

				  \begin{abstract}...\end{abstract}

				  \tableofcontents

				  \chapter*{Document Status Sheet}

\section*{Document Status Overview}
\subsection*{General}
\begin{tabularx}{\linewidth}{@{}lX@{}}
    Document title:     &   \TitleFull \\
    Identification:     &   \TitleAbbr-\Version\\
    Author:             &   \tessa{}, \roel{}, \femke{}, \hugo{} \\
    Document status:    &  Initial \\
\end{tabularx}

\subsection*{Document History}
\begin{tabularx}{\linewidth}{@{}cllX@{}}
    \toprule
    \emph{Version}    &   \emph{Date} & \emph{Author} &  \emph{Reason of change}\\
    \midrule
    0.0 & 3-Jun-2013 & \raggedright{\tessa{}, \roel{}, \femke{}, \hugo{}} & Initial version. \\
    \bottomrule
\end{tabularx}

\section*{Document Change Records Since Previous Issue}
\subsection*{General}
\begin{tabularx}{\linewidth}{lX}
    Date:           &   3-Jun-2013 \\
    Document title: &   \TitleFull \\
    Identification: &   \TitleAbbr-\Version\\
\end{tabularx}

\subsection*{Changes}
\begin{tabular}{lll}
    \toprule
    \emph{Page} & \emph{Paragraph} & \emph{Reason to change} \\
    \midrule
    - & -  & Initial version. \\
    \bottomrule
\end{tabular}

				  \chapter{Introduction}
\label{chap:introduction}
This chapter will explain the purpose of this document as well as what the scope of this document is. That is, what the purpose of this document is and how it is related to other documents in the project.

\section{Purpose}
The purpose of this document is to set rules and guidelines to which all project members should adhere. This will concern the versioning, identification and layout of all documents that are created for this project. All major documents should adhere to strict rules, while for other documents, such as files containing code, guidelines are set that are more loose.

This document should be read as a reference. It can be used when a developer or project member is not sure about how to do something as a mainstay.

\section{Scope}
In this project, the following configuration items (CIs) will be produced:
\begin{itemize}
	\item Architectural Design Document (ADD);
	\item Detailed Design Document (DDD);
	\item Software Configuration Management Plan (SCMP);
	\item Software Project Management Plan (SPMP);
	\item Software Quality Assurance Plan (SQAP);
	\item Software Requirements Document (SRD);
	\item Software Transfer Document (STD);
	\item Software User Manual (SUM);
	\item Software Verification and Validation Plan (SVVP);
	\item User Requirements Document (URD);
	\item Code;
	\item Test plans for a number of phases. In particular:
	\begin{itemize}
		\item Unit Test Plan (UTP);
		\item Integration Test Plan (ITP);
		\item Acceptance Test Plan (ATP).
	\end{itemize}
	\item Product Backlog.
\end{itemize}
The ESA standard mentions a System Test Plan (STP) as well, but in our case this can be omitted.

\section{List of definitions}
\begin{tabular}{@{}ll@{}}
	\toprule
	2IP35 & The Software Engineering Project \\
	ADD   & Architectural Design Document \\
	ATP   & Acceptance Test Plan \\
	BCF   & Bureau for Computer Facilities \\
	CI    & Configuration Item \\
	CM    & Configuration Manager \\
	DDD   & Detailed Design Document \\
	ITP   & Integration Test Plan \\
	SEP   & Software Engineering Project \\
	SCMP  & Software Configuration Management Plan \\
	SPMP  & Software Project Management Plan \\
	SQAP  & Software Quality Assurance Plan \\
	SRD   & Software Requirements Document \\
	STD   & Software Transfer Document \\
	STP   & System Test Plan \\
	SUM   & Software User Manual \\
	SVVP  & Software Verification and Validation Plan \\
	TU/e  & Eindhoven University of Technology \\
	URD   & User Requirements Document \\
	UTP   & Unit Test Plan \\
	\bottomrule
\end{tabular}

\section{List of references}
\bibliography{\pathtobase{ref.bib}}
				  \chapter{Test plan}
\label{chap:testPlan}
In this chapter it is described what items are tested with the acceptance tests, and how these items must be tested. Specific information about each test is described in chapters \ref{chap:testSpecs}-\ref{chap:testReports}.

\section{Test items}
Fingerpaint is designed as described in the ADD \cite{add}. Each component specified there is subject to integration tests as described in this document. These components are: the Client Persistence, the Layout, the Application State, the HTTP Server, the Application Service, the Application Persistence, the Simulator Service and the Fortran Module.

\section{Features to be tested}
As all features require communication between the Layout and the Application State components, all features are tested with the integration tests.

\section{Test deliverables}
Prior to testing, the following documents/code should be completed:
\begin{itemize}
\item ADD \cite{add}
\item ITP \cite{itp}, should be finished up to the test reports (chapter \ref{chap:testReports}).
\item The Fingerpaint code
\end{itemize}
After the tests are concluded the test reports should be written, and problem reports should be written when necessary.

\section{Testing tasks}
Before the integration tests can be executed, the following needs to be done:
\begin{itemize}
\item The integration tests need to be written.
\item Each component needs to be functional and tested.
\item Integration test input data needs to be created.
\end{itemize}

\section{Environmental needs}
The hardware/software required to run \applicationname{} is described in appendix A of the ATP \cite{atp}.

\section{Test case pass/fail criteria}
The integration tests as a whole succeed if all the integration tests in it pass. Similarly if one test fails, the software is rejected.
The test criteria are described in chapter \ref{chap:testSpecs}.

				  ...

				\end{document}
			\end{verbatim}
			\caption{Example of how a document that uses the \texttt{fingerpaint.cls} document class should look like.}
			\label{fig:exampledocument}
		\end{minipage}
	\end{center}
\end{figure}

As noted in section \ref{sec:interfacemanagement}, the CM can assist when project members experience problems with \LaTeX{}.

Finally, all documents are subject to the standards described in the ESA standard \cite{esa} and must also adhere to the requirements as described in the SQAP \cite{sqap} and the SVVP \cite{svvp}.