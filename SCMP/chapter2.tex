\chapter{Management}
\label{chap:management}
In this chapter it is specified which project members are involved in configuration management. Also, the responsibility a team member has in a function involved with configuration management is explained. Then, some general responsibilities are explained. Finally, a template for the creation of documents is given and some conventions are set about document creation, to ensure a consistent layout.

\section{Organisation}
The team members involved in configuration management are the configuration manager (CM) and vice CM. The project members that have volunteered to fulfil these roles are named in the SPMP \cite{spmp}.

Other group members should always assist the CM and vice CM.

\section{Responsibilities}
The CM and vice CM are responsible for copying documents to the master and archive library at the right moments, as mentioned in chapter \ref{chap:configurationcontrol}. They are in general responsible for the contents of the master and archive library. Another task for them is creating and updating document templates, although this task can be delegated.

The CM is primarily responsible for configuration management, although he or she can delegate tasks to the vice CM, in which case the vice CM is responsible. Whenever the CM is (temporarily) not available, the vice CM should take over the tasks of the CM, including the responsibility for these tasks.

Finally, all project members are responsible for the documents they work on. This means that they update the document status sheet and make sure the latest version of the document they work(ed) on is available in the development repository (refer to chapter \ref{chap:configurationcontrol}, section \ref{subsec:configcontrol-library-dev}). When multiple group members work on the same document, they share the responsibility and additionally are responsible for the combined consistency of the document. Also, they should make sure that the repository remains in a ``workable'' state. That is, they should solve possible merge conflicts together.

\section{Interface Management}
\label{sec:interfacemanagement}
The \projectname{} application will be developed using an external virtual server provided by the BCF. In case of failure of this server, the CM will contact BCF and let them resolve the issue. BCF is in this case supposed to have expert knowledge and have the means to resolve issues.

In general, the CM can help other project members when they have trouble with some software that is used (refer to chapter \ref{chap:tools}). However, the CM may delegate this task to other group members who have more expertise on the subject.

\section{SCMP Implementation}
In this project, we will have only one SCMP document, contrary to what is described in the ESA standard \cite{esa}. Thus, this document will not contain a planning for every phase of the project. Instead, refer to the SPMP for the planning of the phases.

\section{Applicable Procedures}
Every non-code document has to be created using \LaTeX{} and should use the \texttt{fingerpaint.cls} document class by declaring the following in the beginning of the document (with other values for the options of course):
\begin{center}\begin{minipage}{0.5\textwidth}\begin{verbatim}
\documentclass[%
    pathtobase=..,%
    titlefull={Full Document Title},%
    titleabbr=FDT,%
    version=0.1]{fingerpaint}
\end{verbatim}\end{minipage}\end{center}
Every document should start with a \verb|\maketitle{}| call, followed by an abstract and then the \verb|\tableofcontents|. This to ensure a consistent style among all documents.

Each document should have a ``main'' \texttt{.tex}-file that \verb|\input|s separate \texttt{.tex}-files that each contain a chapter. This enables project members to work on the same document more efficiently, as working on different chapters will not cause any merge conflicts.

So, a standard ``main'' \texttt{.tex}-file should like as is shown in figure \ref{fig:exampledocument}.

\begin{figure}
	\begin{center}
		\begin{minipage}{0.8\textwidth}
			\begin{verbatim}
				\documentclass[%
				    pathtobase=..,%
				    titlefull={Full Document Title},%
				    titleabbr=FDT,%
				    version=0.1]{fingerpaint}

				\begin{document}

				  \maketitle{}

				  \begin{abstract}...\end{abstract}

				  \tableofcontents

				  \chapter*{Document Status Sheet}

\section*{Document Status Overview}
\subsection*{General}
\begin{tabularx}{\linewidth}{@{}lX@{}}
    Document title:     &   \TitleFull \\
    Identification:     &   \TitleAbbr-\Version\\
    Author:             &   \tessa{}, \lasse{}, \roel{}, \femke{} \\
    Document status:    & Approved by the customer \\
\end{tabularx}

\subsection*{Document History}
\begin{tabularx}{\linewidth}{@{}clXX@{}}
    \toprule
    \emph{Version}    &   \emph{Date} & \emph{Author} &  \emph{Reason of change}\\
    \midrule
    0.0 & 14-June-2013 & \raggedright{\tessa{}, \lasse{}, \roel{}, \femke{}} & Initial version. \\
    0.1 & 18-June-2013 & \raggedright{\tessa{}, \lasse{}} & Fixed inconsistensies between the AT's and the application. \\
    1.0 & 18-June-2013 & - & Approved by the customer. \\
    \bottomrule
\end{tabularx}

\section*{Document Change Records Since Previous Issue}
\subsection*{General}
\begin{tabularx}{\linewidth}{lX}
    Date:           &   18-June-2013 \\
    Document title: &   \TitleFull \\
    Identification: &   \TitleAbbr-\Version\\
\end{tabularx}

\subsection*{Changes}
\begin{tabularx}{\textwidth}{llX}
    \toprule
    \emph{Page} & \emph{Paragraph} & \emph{Reason to change} \\
    \midrule
    - & - & Approved by the customer. \\
    \bottomrule
\end{tabularx}

				  \chapter{Introduction}

\section{Purpose}
   This document describes procedures concerning the testing of the delivered products (product
   documents and software) of the \projectname{} project for compliance with the requirements. The
   requirements that the software has to be verified against can be found in the product documents
   URD \cite{urd}, Product Backlog \cite{backlog}, SRD \cite{srd}, ADD \cite{add} and DDD \cite{ddd}. The modules to be verified and validated are defined
   in the AD phase. The goal of verifying and validating is to check whether the software product to
   be delivered conforms to the requirements of the client and to ensure a minimal number of errors
   in the software. This project document is written for managers and developers of the \projectname{}
   project.

\section{Scope}
The \applicationname{} is an application designed and developed by \projectauthor{} for Prof.dr.ir. P.D. Anderson. The first goal of the \applicationname{} is to provide an intuitive and modern interface for an already existing mixing program. This existing mixing program can calculate how the concentration distribution of a certain mixture changes as the mixer mixes. The second goal of the \applicationname{} is is to provide this service to devices unable to handle the computational load themselves. To this end, the main computation done by the existing mixing program is done on a server. These two goals are formulated as a set of formal requirements in the  URD \cite{urd}.

\section{List of definitions}
\begin{longtable}{l|l}
2IP35 & The Software Engineering Project \\ 
AD    &Architectural Design \\ 
ADD   &Architectural Design Document \\ 
%AT    &Acceptance Test \\ 
ATP   &Acceptance Test Plan \\ 
Client &Prof.dr.ir. P.D. Anderson \\ 
%CM    &Configuration Manager \\ 
CI &Configuration Item\\
DD    &Detailed Design \\ 
DDD   &Detailed Design Document \\ 
ESA   &European Space Agency \\ 
%TU/e  &Eindhoven University of Technology \\ 
%OM    &Operations and Maintenance Plan \\ 
PM    &Project Manager \\ 
QAM &Quality Assurance Manager\\
SCMP  &Software Configuration Management Plan \\ 
SEP   &Software Engineering Project \\ 
%SL    &Software Librarian \\ 
SM	  &Senior Management \\
SPMP  &Software Project Management Plan \\ 
SQAP  &Software Quality Assurance Plan \\ 
%SR    &Software Requirements \\ 
SRD   &Software Requirements Document \\ 
%STD   &Software Transfer Document \\ 
SUM   &Software User Manual \\ 
%SVVP  &Software Verification and Validation Plan \\ 
SVVR  &Software Verification and Validation Report \\ 
%TR    &Transfer phase \\ 
UR    &User Requirements \\ 
URD   &User Requirements Document \\ 
%VPM   &Vice Project Manager \\ 
\end{longtable}

\section{List of references}

\bibliographystyle{plainnat}
\bibliography{../ref}

				  \chapter{Test plan}
\label{chap:testPlan}
In this chapter it is described what is tested and how it is tested with acceptance tests. Specific information about each test is described in chapters \ref{chap:testCaseSpecs}-\ref{chap:testReports}.

\section{Test items}
In the acceptance tests, the requirements as described in the URD\ref{urd} are tested. Specifically, \applicationname{} is tested to see if it fulfills these requirements. In addition, the URD\ref{urd} contains use cases that describe the desired behavior.

\section{Features to be tested}
The features subject to testing are a part of the CPRs described in the URD\ref{urd}. Not all of these requirements are implemented, and only the implemented requirements can be tested. The implemented requirements are: \todo{insert list of implemented requirements}

\section{Test deliverables}
Prior to testing, the following documents/code should be completed:
\begin{itemize}
\item URD\ref{urd}
\item ATP\ref{stp}, should be finished up to the test reports (chapter \ref{chap:testReports}).
\item The Fingerpaint code
\end{itemize}
After the tests are concluded the test reports should be written, and problem reports should be written when necessary.

\section{Testing tasks}
Before the acceptance tests can be executed, the following needs to be done:
\begin{itemize}
\item The acceptance tests need to be written.
\item The \applicationname{} should be functional.
\end{itemize}

\section{Environmental needs}
The hardware/software required to run \applicationname{} is described in appendix A of the ATP\ref{atp}.

\section{Test case pass/fail criteria}
The acceptance tests as a whole succeed if all the acceptance tests in it pass. Similarly if one test fails, the software is rejected.
The test criteria are described in chapter \ref{chap:testSpecs}.
				  ...

				\end{document}
			\end{verbatim}
			\caption{Example of how a document that uses the \texttt{fingerpaint.cls} document class should look like.}
			\label{fig:exampledocument}
		\end{minipage}
	\end{center}
\end{figure}

As noted in section \ref{sec:interfacemanagement}, the CM can assist when project members experience problems with \LaTeX{}.

Finally, all documents are subject to the standards described in the ESA standard \cite{esa} and must also adhere to the requirements as described in the SQAP \cite{sqap} and the SVVP \cite{svvp}.