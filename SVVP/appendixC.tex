\chapter{Sprints phase} \label{Sprints phase}

\section{SR phase}
\subsection{The Software Requirements Review}
The outputs of the Software Requirements Definition Phase are formally reviewed in the Software Requirements Review (SR-R). Internal reviews are held before the formal external SR-R takes place. It ensures that:
\begin{itemize}
	\item The SRD states the software requirements clearly, completely and in sufficient detail to start the design phase.
\end{itemize}

\subsection{Requirements for Software Requirements}
In the SRD each requirement must:
\begin{itemize}
	\item have a unique identifier. Traceability in other phases depends on this identifier.
	\item be marked essential or not. Essential requirements have to be met for the software to be acceptable.
	\item have a priority if the transfer will proceed in phases.
	\item be marked if unstable. Unstable requirements may depend on feedback from other phases.
	\item have references that trace it back to the URD. A software requirement is caused by one or more user requirements.
	\item be verifiable.
\end{itemize}

Besides the requirements the SRD must contain a traceability matrix containing trace from the software requirements to the user requirements and vice-versa. This matrix should be complete, meaning all requirements should be traceable.

\section{AD phase}
\subsection{The Architectural Design Review}
The outputs of the Architectural Design phase are formally reviewed in the Architectural Design Review (AD-R). Any report by the QAM may also be input for the AD-R. Internal reviews are held before the formal external AD-R takes place. It ensures that:
\begin{itemize}
	\item The ADD describes the optimal solution to the problem stated in the SRD.
	\item The ADD describes the architectural design clearly, completely and in sufficient detail to start the detailed design phase.
	\item  The ITP is an adequate plan for integration testing the software in the DD phase.
\end{itemize}

\subsection{Design Quality}
A good design is:
\begin{itemize}
	\item Adaptable: it is easy to modify and maintain.
	\item  Efficient: it makes a minimal use of available resources.
	\item  Understandable: it is not only clear for the developers, but also for outsiders.
	\item  Modular: the components are simple and independent from each other:
	\begin{itemize}
		\item  A change in one component has minimal impact on other components.
		\item A small change in requirements does not lead to system wide changes.
		\item  The effects of an error condition are isolated to its source component.
		\item A component is understandable as a stand-alone unit, without reference to others.
	\end{itemize}
\end{itemize}

Good components obey the information hiding principle: software design decisions are encapsulated so that the interface reveals as little as possible about its inner workings. For example, a component should hide how its data is stored: it can be in memory (in an array, list, tree, ...) or in a temporary file.
Some rules to choose components:
\begin{itemize}
	\item Minimize coupling between components.
	\begin{itemize}
		\item Minimize the number of items that are passed between components;
		\item Pass only the data that are needed (data coupling);
		\item Do not pass a structure of which only a small part is being used (stamp coupling);
		\item Avoid the use of control flags (control coupling);
		\item Do not use global data (common coupling).
	\end{itemize}
	\item Maximize cohesion inside a component: put elements into a component that are related to each other; they contribute for example to the same task.
	\item Restrict fan-out: restrict the number of child components.
	\item Maximize fan-in: reuse components as often as possible.
	\item Use factoring: avoid duplication of functionality. Cut the common functionality from the components and put it into a reusable component.
	\item Separate logical and physical functionality: top-level components must be separated from physical aspects (the data they deal with); the level of abstraction of a component must be according to its place in the hierarchy.
\end{itemize}

\section{DD phase}
\subsection{The Detailed Design and Production Review}
The outputs of the Detailed Design and Production Phase are formally reviewed in the Detailed Design and Production Review (DD-R). Any report of the QAM may also be input for the DD-R. Internal reviews are held before the formal external DD-R takes place. It ensures that:
\begin{itemize}
	\item The DDD describes the detailed design clearly, completely and in sufficient detail to allow development and maintenance by software engineers not involved in the project.
	\item  Modules have been coded according to the DDD.
	\item Modules have been verified according to the unit test specifications in the UTP.
	\item Major components have been integrated according to the ADD.
	\item Major components have been verified according to the integration test specifications in the ITP.
	\item The ATP specifies the test design, test procedures and test cases so that all user requirements in the URD can be validated.
\end{itemize}