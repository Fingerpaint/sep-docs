\documentclass[%
		pathtobase=..,%
		titlefull={Software Validation and Verification Plan},%
		titleabbr=SVVP,%
		version=0.1]{fingerpaint}

\begin{document}

\maketitle

\begin{abstract}
This is the Software Validation and Verification Plan (SVVP) of the \projectname{} project, developed in the context of
the Software Engineering Project (2IP35). This document contains the procedures for verification and validation and complies with the Software Engineering Standard as specified by the European Space Agency (ESA) \cite{esa}.
\end{abstract}

\tableofcontents

\chapter*{Document Status Sheet}

\section*{Document Status Overview}
\subsection*{General}
\begin{tabularx}{\linewidth}{@{}lX@{}}
    Document title:     &   \TitleFull \\
    Identification:     &   \TitleAbbr-\Version\\
    Author:             &   \tessa{}, \roel{}, \femke{}, \hugo{} \\
    Document status:    &  Initial \\
\end{tabularx}

\subsection*{Document History}
\begin{tabularx}{\linewidth}{@{}cllX@{}}
    \toprule
    \emph{Version}    &   \emph{Date} & \emph{Author} &  \emph{Reason of change}\\
    \midrule
    0.0 & 3-Jun-2013 & \raggedright{\tessa{}, \roel{}, \femke{}, \hugo{}} & Initial version. \\
    \bottomrule
\end{tabularx}

\section*{Document Change Records Since Previous Issue}
\subsection*{General}
\begin{tabularx}{\linewidth}{lX}
    Date:           &   3-Jun-2013 \\
    Document title: &   \TitleFull \\
    Identification: &   \TitleAbbr-\Version\\
\end{tabularx}

\subsection*{Changes}
\begin{tabular}{lll}
    \toprule
    \emph{Page} & \emph{Paragraph} & \emph{Reason to change} \\
    \midrule
    - & -  & Initial version. \\
    \bottomrule
\end{tabular}


\chapter{Introduction}

\section{Purpose}
   This document describes procedures concerning the testing of the delivered products (product
   documents and software) of the \projectname project for compliance with the requirements. The
   requirements that the software has to be verified against can be found in the product documents
   URD \cite{urd}, Product Backlog \cite{backlog}, SRD \cite{srd}, ADD \cite{add} and DDD \cite{ddd}. The modules to be verified and validated are defined
   in the AD phase. The goal of verifying and validating is to check whether the software product to
   be delivered conforms to the requirements of the client and to ensure a minimal number of errors
   in the software. This project document is written for managers and developers of the \projectname
   project.

\section{Scope}
\todo{Summary of products to be verified and specifications to be verified against.}
\section{List of definitions}
\begin{tabular}{l|l}
2IP35 & The Software Engineering Course \\ 
AD    &Architectural Design \\ 
ADD   &Architectural Design Document \\ 
AT    &Acceptance Test \\ 
ATP   &Acceptance Test Plan \\ 
Client & \todo{The client} \\ 
CM    &Configuration Manager \\ 
DD    &Detailed Design \\ 
DDD   &Detailed Design Document \\ 
ESA   &European Space Agency \\ 
TU/e  &Eindhoven University of Technology \\ 
OM    &Operations and Maintenance Plan \\ 
PM    &Project Manager \\ 
QM    &Quality Manager \\ 
SCMP  &Software Configuration Management Plan \\ 
SEP   &Software Engineering Project \\ 
SL    &Software Librarian \\ 
SM	  &Senior Management \\
SPMP  &Software Project Management Plan \\ 
SQAP  &Software Quality Assurance Plan \\ 
SR    &Software Requirements \\ 
SRD   &Software Requirements Document \\ 
STD   &Software Transfer Document \\ 
SUM   &Software User Manual \\ 
SVVP  &Software Verification and Validation Plan \\ 
SVVR  &Software Verification and Validation Report \\ 
TR    &Transfer phase \\ 
UR    &User Requirements \\ 
URD   &User Requirements Document \\ 
VPM   &Vice Project Manager \\ 
\end{tabular}
\section{List of references}

\bibliographystyle{plainnat}
\bibliography{../ref}

\chapter{Verification overview}
\section{Organization}
\subsection{Organization}
   The QAM checks the verification and validation of the activities of the project. Therefore
   the QAM attends every internal or external review. If the QAM is not available the vice-QAM
   will take his place, this means that every time the QAM is mentioned it can also be the vice-QAM.
   If the QAM runs into problems he reports them to the PM. The PM needs to verify that these
   problems are resolved.
       The project uses the following methods of verification and validation:
\section{Internal reviews} \label{internal review}
   In order to keep the quality of our documents up to standards they will be subject to internal
   reviews.
       The team carrying out the internal review of a technical or management document will at least
   consist of the following persons:
\begin{itemize}
\item The QAM. He will make the review document.
\item One of the authors of the document.
\item At least one other member of the project team, not part of the authoring team.
\item The adviser/PM may also be present if necessary.
\end{itemize}
More details about internal reviews can be found in Section \ref{internal review details}.
\section{External reviews} \label{external review}
When a document has been internally accepted it should have the desired quality. Having the right amount of quality does not automatically mean that the document conforms to the customers expectations. Therefore an external review is held. The documents which need external reviews are the URD, Product Backlog, SRD, ATP, SUM and ADD.

An external review can only take place after the document has been approved by the adviser. Any documents sent to the adviser will have to be accepted internally first. The external reviews of management documents will be done by the SM. The team carrying out the external review of a technical document will consist of the following people:
\begin{itemize}
\item The adviser (if available).
\item At least one the author(s) (of the document to be reviewed).
\item The QAM. He will make the review document.
\item At least one other member of the team.
\item When necessary (URD \cite{urd}, Product Backlog \cite{backlog}, SRD \cite{srd}, ATP \cite{atp} and SUM \cite{sum}) also the customer
\end{itemize}
More details about exteranl reviews can be found in Section \ref{external review details}.
\section{Audits}
    Audits are reviews that assess compliance with software requirements, specifications, baselines,
    standards, procedures, instructions, codes and licensing requirements. Physical audits check that
    all items identified as being part of the configuration are present in the product baseline. A
    functional audit checks that unit tests, integration tests and system tests have been carried out
    and records their success or failure. Functional and physical audits can be performed before
    the release of the software (ESA Software Engineering Standard \cite{esa}). The SM is
    allowed to audit the project to check if the procedures, as described in the management
    documents SPMP \cite{spmp}, SQAP \cite{sqap}, SCMP \cite{scmp} and this document are followed. Audits are not routine
    checks, but the SM can request them.
        The following rules apply to all audits:
\begin{itemize}
\item Only the SM can request audits.
\item Audit requests must be directed to the PM.
\item In the audit request the following information must be included:
\begin{itemize}
\item Names of the auditors (at least two persons)
\item Possible dates of the audit
\item Purpose of the audit
\item Items that will be checked
\end{itemize}
\item The audit is attended by a the QAM, the PM and possibly others as indicated by SM.
\item The results of the audit are reported by a group member in a written report to the PM and the QAM within one week of the audit. This report must contain the following information:

	\begin{itemize}
	\item Date of the audit
	\item Participants in the audit
	\item Checked items
	\item Conclusion
	\item Recommendations
	\end{itemize}
\end{itemize}


\section{Tests}
At all times, tests covering functionality are coded before that functionality is implemented. 
All tests covering all (impelemented) functionality together comprise the test suite, or regression tests.
Before and after each code change, the test suite is run, and should pass before code is changed or committed to version control. This follows and extends on Test-Driven Development.

 For each type of testing there is a separate test plan. 
Note that these test plans are updated iteratively each sprint. The following test plans (plans that outline the approach to testing) can be found as separate documents:

\begin{itemize}
\item ATP (Acceptance Test Plan) \cite{atp}
\item STP (System Test Plan) \cite{stp}
\item ITP (Integration Test Plan) \cite{itp}
\item UTP (Unit Test Plan) \cite{utp}
\end{itemize}

The ATP \cite{atp} has to be approved by the client, as it will define the terms on which the final product will be accepted. The results of the tests are presented to the PM and the QAM. The produced code and product documents must also be tested to assure that all the requirements are met. This can be found in Section \ref{tracing} and is documented in the appendix \ref{Sprints phase}.

\section{Schedule}
The schedules for all phases are given in the SPMP \cite{spmp}.
\section{Resources}
In this project no special tools or people are used for verifing documents.
\section{Project responsibilities}
    Some roles as defined in the SPMP have responsibilities related to verification and validation.
    These responsibilities are:\\

\textbf{Member of development team:}
\begin{itemize}
\item The work is adequately inspected.
\end{itemize}

\textbf{Quality Assurance Manager:}
\begin{itemize}
\item Assuring that the requirements of the documents are adhered to.
\item Assuring the documents conform to the specified layout and contain the proper information.
\item Lead the review sessions.
\item Manage the test runs.
\end{itemize}

\textbf{Configuration Manager:}
\begin{itemize}
\item Tag documents that have been committed for review, and the approved versions if changes where needed.
\end{itemize}

\textbf{Scrum Master:}
\begin{itemize}
\item Ensure that the Scrum process is followed.
\item Check that the backlog is updated and that stories are clear.
\end{itemize}

\textbf{Product Owner:}
\begin{itemize}
\item Check that the items in the product backlog are user centered rather than technical.
\end{itemize}


\section{Tools, techniques and methods}
The tools that are used during the project are discussed in the SCMP \cite{scmp}.
\chapter{Administrative procedures}
\section{Anomaly reporting and resolution}
Everything that is not up to standards it should be up to or does not conform to requirements it should conform to, is an anomaly. Procedures for anomaly resolution can be found in the SQAP \cite{sqap}. Furthermore, it is the task of the SQA team to monitor whether the procedures as defined in the management plans (SPMP \cite{spmp}, SCMP \cite{scmp}, SQAP \cite{sqap} and SVVP) are followed. This is done during team meetings, reviews and by randomly checking CIs. Findings are reported to the PM. It then is the responsibility of the PM to enforce compliance with defined procedures. If the results of the PMs actions are not satisfactory to the QAM, he can request the senior management to take further action.

\section{Task iteration policy}
    Every task performed is to be interally reviewed as described in Chapter \ref{verificationactivities}. Some tasks (see Section \ref{external review}) need an external review. \bibliographystyle{} If, during a review, problems
    are discovered concerning the correct conclusion of the task a decision is made concerning the
    iteration of the task. Guidelines are provided for the following cases:
\begin{itemize}
\item The team responsible was unable to complete their task, most probably because of one of
          the risks as described in Section 3.3 of the SPMP \cite{spmp}. In this case, it is the responsibility
          of the QAM to solve the problem and make sure the task is completed as described in the
          SPMP. If the QAM is unable to do so, he must report this to the PM. If problems arise
          concerning the dependencies between tasks these are to be reported to the PM.
\item A structural error was found in the execution of the task, for example the output of a piece
          of code that does not comply with the requirements.
          In this case, the team that is responsible performs the task again. If necessary the PM
          schedules extra manhours.
\item An item was forgotten during the execution of a task.
          Depending on the severity of this item the QAM will decide whether a redo of the task is
          needed, or only the forgotten item needs to be fixed. This case will most probably occur in
          processing review remarks.
\end {itemize}
\section{Deviation policy}
    During the project, the procedures described in the management documents
    are followed. However, if in the QAM’s opinion, this endangers the completion of the
    project then the QAM can decide to deviate from these procedures. If the decision is made
    to deviate from the procedures described in the management documents, the PM must be
    informed of such a deviation.

\section{Control procedures}
Procedures assuring that configuration items are not accidentally or deliberately changed are
described in the SCMP \cite{scmp}.

\section{Standards}
Before both internal and external reviews, the authors certify that the document is according to
ESA Software Engineering standard \cite{esa}, and that the document complies with the standard layout
as detailed in SCMP \cite{scmp}.

\chapter{Verification activities} \label{verificationactivities}
\section{Reviews}
    Review procedures are held during all phases of the \projectname project. Configuration items are reviewed in the phase they are delivered; an overview of which item is delivered in which phase can be found in the SPMP \cite{spmp}. All project and product documents have one of the following statuses:
\begin{itemize}
\item Draft (initial status)
\item Internally approved with proposed changes
\item Internally approved
\item Conditionally approved
\item (Externally) approved
\end{itemize}
Note that approved technical documents are not modified (unless the completion of the project is endangered). With respect to the approved management documents only appendices for every phase are added during the project. The appendices are approved during review meetings.

For a document to become (internally) approved it has to be reviewed. Here internal and external reviews of technical and management documents (Section \ref{internal review} and \ref{external review} respectively) are distinguished.

As noted above each document starts with the draft status. Once there has been a internal review it either becomes Internally approved or needs some changes. When there are only minor changes needed the document will be internally approved when the QAM has confirmed that the changes have been made. If it concerns major changes a new internal review will be needed.

Internally approved documents can be scheduled for external review. During this review the document can reach the highest status of externally approved if there are no defects. If there are only minor defects the document may be conditionally approved, these defects need to be solved to retrieve the highest status. If major defects show up during the review the document needs to be changed to solve these defects. Because this involves great changes it should pass a new internal review before it may be subject to another external review.

\subsection{Internal reviews} \label{internal review details}
The QA is the member of the SQA-team present. T is the time of the review meeting.


\begin{tabular}{|l|l|p{20em}|l|}
\hline
 Nr &Actor     &  Action                                                 & Time \\
\hline
 1  & QAM/QA   & Set a date for the internal review of the document      &    -\\
 2  & Leader   &  Deliver the paper review version of the document to the reviewers   & T - 2 workdays\\
                
 3  & Reviewer &  Inspect the document (language errors are underlined)  & Before T\\
 4  & Reviewer &  Discuss all errors other than language errors          & T\\
 5  & Leader   &  Write down all necessary changes                       & T\\
 6  & Reviewer &  Decide if the document can be approved, provided the stated changes are made  & T\\
                
 7  & QA       &  If the document cannot be approved, an appointment for a new review meeting is made    & T\\
                
 8  & Leader   &  Collect annotated documents                            & T\\
 9  & QAM/QA   & See to it that the stated remarks are handled properly by the team delivering the document  &   After T\\
                
 10 & QAM/QA   & Grant the document the status internally accepted if all requested changes are made     &   After T\\
                
\hline
\end{tabular}

\subsection{External reviews} \label{external review details}
    For the organization of external reviews see Section \ref{external review}. The following table shows the action list for
    the preparation and execution of the external reviews of documents. T is the time of the review
    meeting. This procedure is only for the external review of product documents. The metrics of
    the external review will be sent to the SM. The official format for reviews is described in Appendix \ref{format review}.

\begin{tabular}{|l|p{6em}|p{20em}|p{6em}|}
\hline
Nr. &Actor &     Action       &                                                Time\\
\hline
1  & QAM   &      Set a date and place for the external review of the document & After internal acceptance\\
2  & Author    & Deliver the paper version of the document to all reviewers  & T - 5 workdays\\
3  & Reviewer &Inspect the document and write down all errors explicitly     & Before T\\
4  & Reviewer &Deliver remarks to the moderator                              & Before T\\
5  & QAM      &  Inspect remarks                                             & Before T\\
6  & Author   &  Lead the meeting and keep discussions to the point          & T\\
7  & QAM      &  Document everything that is discussed during the review     & T\\
8  & Reviewer &Discuss all comments that need explanation or discussion     &  T\\
9  & Author   &  Collect the remarks on the documents                       &  After T\\
10 & Reviewer, QAM & Decide the status of the document at the end of the meeting. There are three possible outcomes: the document is rejected and a new appointment is made the document is accepted and the status Approved is granted  & After T\\
11 & QAM     &  Make minutes of the review, and hand these together with the remarks of the reviewers to the Senior Management. Also make sure they will go to the configuration management system  & After T\\
\hline               
\end{tabular}

Only when the document is rejected do actions 12 and 13 apply.\\

\begin{tabular}{|l|p{6em}|p{20em}|p{6em}|}    
\hline          
12 & QAM  &      See to it that the remarks are handled properly by the team responsible for the document & After T\\
               
13 & QAM  &      Grant the document the status Approved if all reviewers inform that their remarks are handled properly, eventually       after another review if the remarks included major changes & After T\\
\hline
\end{tabular}


\section{Formal proofs}
Formal proof will be given where considered necessary by the SQA team, or when asked by the person(s) responsible for a certain product.

\section{Tracing} \label{tracing}
During the project the relation between the input and the output of a phase must be checked several times. A traceability table as result of the final trace is included in the output document of the phase. In this table the CI is traced to the input of the phase. During the software life cycle it is necessary to trace:

\begin{itemize}
\item User requirements to software requirements and vice versa, this is documented in the SVVP (Appendix \ref{Sprints phase}).
\item Software requirements to component requirements and vice versa, this is documented in the SVVP (Appendix \ref{Sprints phase}).
\item Component requirements to DD requirements and vice versa, this is documented in the SVVP (Appendix \ref{Sprints phase}).
\item Integration tests to architectural units and vice versa, this is described in the integration test plans \cite{itp}. These tests are performed during the sprints.
\item Unit tests to the modules of the detailed design, this is described in the unit test plans \cite{utp}. These tests are performed during the sprints.
\item System tests to software requirements and vice versa, this is described in the system test plans \cite{stp}. These plans \cite{spmp} are executed during the sprints.
\item Acceptance tests to user requirements and vice versa, this is described in the acceptance test plans \cite{atp}. These tests are executed during the sprints.
\end{itemize}
To support traceability, all requirements are uniquely identified.

\chapter{Verification reporting}
    For the verification and validation of technical CIs (apart from the URD \cite{urd}) two parts are
    added to these CIs:
\begin{itemize}
\item A verification report
\item A validation report
\end{itemize}
    These reports are presented to and checked by a member of the SQA team. The people
    performing the test of the CI write the verification report. The people who delivered the CI
    write the validation report. These are both checked when the CI is reviewed. A validation
    report is written as a result of the tracing. It contains a traceability table. A verification
    report is written as a result of a test. It contains the following items:
\begin{itemize}
\item Unique reference number of the test plan
\item Problems discovered and, if available, solutions to these
\item Acceptance or disapproval of the CI. In case of disapproval, accompanied with a short explanation of the reasons of disapproval
\end{itemize}
For the verification and validation of the entire \projectname project, progress meetings are held with the SM according to the SPMP \cite{spmp}.


\appendix
\chapter{Format of reviews}\label{format review}
\todo{Describe a format for the reviews}


\chapter{User Requirements phase} \label{UR phase}
\section{The User Requirements Review}
The outputs of the User Requirements Definition Phase are formally internally and externally reviewed in the User Requirements Review (UR/R). It ensures that the URD/Product Backlog states the user requirements clearly and completely and a general description of the processes to be supported (the environment) is present. The SPMP, SCMP, SVVP and SQAP are only internally reviewed.

\section{Requirements for user requirements}
User requirements (written as stories) should be realistic, that is:
\begin{itemize}
\item Clear.
\item Verifiable.\\
	``The product shall be user friendly'' is not verifiable.
\item Complete.
\item Accurate.\\
          Among other things, the URD/Product Backlog is inaccurate if it requests something that the user does not
          need, for example a superfluous capacity.
\item Feasible.
\item Traceable, i.e. every requirement should have an unique identifier.
\end{itemize}

\chapter{Sprints phase} \label{Sprints phase}



\end{document}
