\chapter{Administrative procedures}
\section{Anomaly reporting and resolution}
Everything that is not up to standards it should be up to or does not conform to requirements it should conform to, is an anomaly. Procedures for anomaly resolution can be found in the SQAP \cite{sqap}. Furthermore, it is the task of the SQA team to monitor whether the procedures as defined in the management plans (SPMP \cite{spmp}, SCMP \cite{scmp}, SQAP \cite{sqap} and SVVP) are followed. This is done during team meetings, reviews and by randomly checking CIs. Findings are reported to the PM. It then is the responsibility of the PM to enforce compliance with defined procedures. If the results of the PMs actions are not satisfactory to the QAM, he can request the senior management to take further action.

\section{Task iteration policy}
    Every task performed is to be interally reviewed as described in Chapter \ref{verificationactivities}. Some tasks (see Section \ref{external review}) need an external review. \bibliographystyle{} If, during a review, problems
    are discovered concerning the correct conclusion of the task a decision is made concerning the
    iteration of the task. Guidelines are provided for the following cases:
\begin{itemize}
\item The team responsible was unable to complete their task, most probably because of one of
          the risks as described in Section 3.3 of the SPMP \cite{spmp}. In this case, it is the responsibility
          of the QAM to solve the problem and make sure the task is completed as described in the
          SPMP. If the QAM is unable to do so, he must report this to the PM. If problems arise
          concerning the dependencies between tasks these are to be reported to the PM.
\item A structural error was found in the execution of the task, for example the output of a piece
          of code that does not comply with the requirements.
          In this case, the team that is responsible performs the task again. If necessary the PM
          schedules extra manhours.
\item An item was forgotten during the execution of a task.
          Depending on the severity of this item the QAM will decide whether a redo of the task is
          needed, or only the forgotten item needs to be fixed. This case will most probably occur in
          processing review remarks.
\end {itemize}
\section{Deviation policy}
    During the project, the procedures described in the management documents
    are followed. However, if in the QAM’s opinion, this endangers the completion of the
    project then the QAM can decide to deviate from these procedures. If the decision is made
    to deviate from the procedures described in the management documents, the PM must be
    informed of such a deviation.

\section{Control procedures}
Procedures assuring that configuration items are not accidentally or deliberately changed are
described in the SCMP \cite{scmp}.

\section{Standards}
Before both internal and external reviews, the authors certify that the document is according to
ESA Software Engineering standard \cite{esa}, and that the document complies with the standard layout
as detailed in SCMP \cite{scmp}.
