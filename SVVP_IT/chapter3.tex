\chapter{Test case specifications}
In this chapter we will specify the different test cases that verify the correctness of the communication between the different components described in \ref{ADD-fig:compdependencies}. All testcases can be found in the \textbf{Fingerpaint/test.src/nl.tue.fingerpaint.*} package, where * is the name of one of the sections below.
%Clarification: We don't have any integration tests, but some jUnit tests make use of several components. Thus they can serve as integration tests.

%TODO: We verwijzen momenteel naar compdependencies.pdf in ADD; Is dit het beste, of misschien vervangen met SwoftwareTiers.pdf uit het SRD?

\section{client}
\begin{tabular}{llrl}
\emph{ID} & \emph{Test name} & \emph{Integrated Components} & \\
\hline
ITC1 & ApplicationStateTest.java & Layout - & Application State \\
& & Application State - & Client Persistence \\
& & Application State - & HTTP Server \\
& & HTTP Server - & Application Service \\
& & Application Service - & Application Persistence\\
& & Application Service - & Simulator Service\\
& & Simulator Service - & Fortran Module \\
\end{tabular}

\section{client.simulator}
\begin{tabular}{llrl}
\emph{ID} & \emph{Test name} & \emph{Integrated Components} & \\
\hline
ITC2 & SimulatorServiceTest.java & Application State - & HTTP Server \\
& & HTTP Server - & Application Service \\
& & Application Service - & Simulator Service \\
& & Simulator Service - & Fortran Module\\
\end{tabular}

\section{client.storage}
\begin{tabular}{llrl}
\emph{ID} & \emph{Test name} & \emph{Integrated Components} & \\
\hline
ITC3 & StorageManagerTest.java & Application State - & Client Persistence \\
\end{tabular}


\section{server.simulator}
\begin{tabular}{llrl}
\emph{ID} & \emph{Test name} & \emph{Integrated Components} & \\
\hline
ITC4 & SimulatorServiceTest.java & Application Service - & Simulator Service \\
& & Simulator Service - & Fortran Module\\
\end{tabular}

%\section{util}  %Left out because there are no testcases that use any integration
%\section{suits} %Left out because there are no testcases that use any integration


% For each test case, a section with the following structure:
%\section{Name of the test case}
%
%\subsection{Test case identifier}
%\todo{A unique identifier}
%
%\subsection{Test items}
%\todo{The items to be tested}
%
%\subsection{Input specifications}
%\todo{Input for this test case}
%
%\subsection{Output specifications}
%\todo{Output required from this test case}
%
%\subsection{Environmental needs}
%\todo{The test environment}


