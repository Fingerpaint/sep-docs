\chapter{Test plan}
\label{chap:testPlan}
In this chapter it is described what is tested and how it is tested with integration tests. Specific information about each test is described in chapters \ref{chap:testCaseSpecs}-\ref{chap:testReports}.

\section{Test items}
Fingerpaint is designed as described in the ADD\ref{add}. Each component specified there is subject to integration tests a described in this document. These components are: the Application State, the Client Persistence, the Layout, the Update Application State,the HTTP Server, the Application Service, the Application Persistence, the Simulator Service and the Fortran Module.

\section{Features to be tested}
As all features require communication between the Layout and the Application State components, all features are tested with the integration tests.

\section{Test deliverables}
Prior to testing, the following documents/code should be completed:
\begin{itemize}
\item ADD\ref{add}
\item ITP\ref{itp}, should be finished up to the test reports (chapter \ref{chap:testReports}).
\item The Fingerpaint code
\end{itemize}
After the tests are concluded the test reports should be written, and problem reports should be written when necessary.

\section{Testing tasks}
Before the integration tests can be executed, the following needs to be done:
\begin{itemize}
\item The integration tests need to be written.
\item Each component needs to be functional and tested.
\item Integration test input data needs to be created.
\end{itemize}

\section{Environmental needs}
The hardware/software required to run \applicationname{} is described in appendix A of the ATP\ref{atp}.

\section{Test case pass/fail criteria}
The integration tests as a whole succeed if all the integration tests in it pass. Similarly if one test fails, the software is rejected.
The test criteria are described in chapter \ref{chap:testCaseSpecs}.
