\chapter{Test procedures}\label{chap:testProc}
This chapter describes the procedures that have to be followed when writing and executing unit tests. Some general procedures that are applicable to server and client side testing are discussed in section \ref{sec:genProc}. The specific procedures that have to be adhered for client side testing can be found in section \ref{sec:clientProc}, and the procedures that have to be adhered for client side testing can be found in section \ref{sec:serverProc}.\\
 A procedure on how to execute these tests is given in section \ref{sec:exectest}.

\section{General procedures}\label{sec:genProc}
For both client and server side functionality, the author of the code of a certain component should also write test cases accordingly for this component. It should never be the case that code is committed before it has been tested. Moreover, whenever existing functionality is re-factored, the person who changes the code must make sure that all the test cases still pass after the re-factoring. \\
As a rule of thumb, each class from the \texttt{src} folder should have its own test class in the \texttt{test} folder. The name of the test class should clearly indicate the test class under consideration and should be suffixed with \texttt{Test}. The test classes should only test functionality of the class under consideration. However, it might be necessary to include functionality of other classes (such as instantiating other classes) when testing a particular class.
If a new testcase has been written, it should be added to one of the testsuites present in the code, in order to enable automatic testing from the command line.

\section{Client side testing}\label{sec:clientProc}
On the client side, functionality can be tested with either GWT JUnit tests or with Selenium tests. The GWT JUnit tests can be used whenever a specific component has to be tested without the use of a graphical user interface. For example, tests regarding saving to and loading from the local storage can be tested internally: this can be tested by calling the appropriate store and retrieve functionality from the storage-related classes. Testing via the graphical user interface can be achieved through Selenium tests. These tests are written in Java and can be executed on several browsers of choice. After running the tests, the results for the selected browsers can be compared automatically, using a screen-shot comparator.
All the GWT JUnit tests can be executed from two test suites, which can be found at \texttt{ClientSuite.java} and \texttt{SharedSuite.java}. In a similar fashion, all Selenium tests can be executed via \texttt{SeleniumSuite.java}. \\
As mentioned before, some special settings are required to run the unit tests. This also applies to the test suites, as these contain all the individual unit tests regarding specific functionality. All the unit tests in the test suites will only pass, when they are executed in manual mode.

\section{Server side testing}\label{sec:serverProc}
Server side testing can be done through JUnit tests, in a similar way as the GWT JUnit tests in section \ref{sec:clientProc}. Again, a test suite is available to execute all the server tests at once: \texttt{ServerSuite.java}.

\section{Executing the tests}\label{sec:exectest}
To execute the tests, it is assumed the environment of the system is setup correctly as described in ATP appendix \ref*{ATP-environment_setup}. 
Assuming the application is present in \textsc{<app-root>}, navigate the command-line to \textsc{<app-root>/fingerpaint}. To run the tests, one
can now execute the following command:

\begin{verbatim}
ant test
\end{verbatim}

This command compiles the application. After it is finished, a URL is presented on the command-line. Navigate the browser you would like to use to run the tests to this URL. The GWTTestCases wil now be run in this browser. Selenium tests and standard Junin tests will also be automatically run. After all tests are finished, a test report will be printed on the command line.

% For each test procedure, a section with the following structure:
%\section{Name of the test procedure}

%\subsection{Test procedure identifier}
%\todo{A unique identifier}

%\subsection{Purpose}
%\todo{The purpose of this test procedure and the test cases this procedure executes}

%\subsection{Procedure steps}
%\todo{How to log, set up, start, proceed, measure, shut down, restart, stop the test}