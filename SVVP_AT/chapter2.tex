\chapter{Test plan}
\label{chap:testPlan}
In this chapter it is described what is tested and how it is tested with acceptance tests. Specific information about each test is described in chapters \ref{chap:testCaseSpecs}-\ref{chap:testReports}.

\section{Test items}
In the acceptance tests, the requirements as described in the URD\ref{urd} are tested. Specifically, \applicationname{} is tested to see if it fulfills these requirements. In addition, the URD\ref{urd} contains use cases that describe the desired behavior.

\section{Features to be tested}
The features subject to testing are a part of the CPRs described in the URD\ref{urd}. Not all of these requirements are implemented, and only the implemented requirements can be tested. The implemented requirements are: \todo{insert list of implemented requirements}

\section{Test deliverables}
Prior to testing, the following documents/code should be completed:
\begin{itemize}
\item URD\ref{urd}
\item ATP\ref{stp}, should be finished up to the test reports (chapter \ref{chap:testReports}).
\item The Fingerpaint code
\end{itemize}
After the tests are concluded the test reports should be written, and problem reports should be written when necessary.

\section{Testing tasks}
Before the acceptance tests can be executed, the following needs to be done:
\begin{itemize}
\item The acceptance tests need to be written.
\item The \applicationname{} should be functional.
\end{itemize}

\section{Environmental needs}
The hardware/software required to run \applicationname{} is described in appendix A of the ATP\ref{atp}.

\section{Test case pass/fail criteria}
The acceptance tests as a whole succeed if all the acceptance tests in it pass. Similarly if one test fails, the software is rejected.
The test criteria are described in chapter \ref{chap:testSpecs}.