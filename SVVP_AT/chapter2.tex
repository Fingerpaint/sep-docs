\chapter{Test plan}
\label{chap:testPlan}
This chapter describes what items are tested with the acceptance tests, and how these items must be tested. Specific information about each test can be found in chapters \ref{chap:testCaseSpecs}-\ref{chap:testReports}.

\section{Test items}
In the acceptance tests, the \applicationname{} is tested to see if it fulfills the requirements as described in the URD \cite{urd}. In addition, the URD \cite{urd} contains use cases that describe the desired behavior.

\section{Features to be tested}
The features subject to testing are a part of the CPRs described in the URD \cite{urd}. Not all of these requirements are implemented, and only the implemented requirements can be tested. The implemented requirements are: \textbf{CPR1, CPR2, CPR6, CPR7, CPR8, CPR9, CPR10, CPR11, CPR12, CPR13, CPR17, CPR18, CPR19, CPR20, CPR21, CPR22, CPR23, CPR24, CPR29, CPR30, CPR31, CPR32, CPR33, CPR34, CPR35, CPR36, CPR37 and CPR40}.

\section{Test deliverables}
Prior to testing, the following documents/code should be completed:
\begin{itemize}
\item URD \cite{urd}
\item ATP \cite{atp}, should be finished up to the test reports (chapter \ref{chap:testReports}).
\item The Fingerpaint code
\end{itemize}
After the tests are concluded, test reports should be written. Problem reports should be written when necessary.

\section{Testing tasks}
Before the acceptance tests can be executed, the following needs to be done:
\begin{itemize}
\item The acceptance tests need to be written.
\item The \applicationname{} should be functional.
\end{itemize}

\section{Environmental needs}
The hardware/software required to run the \applicationname{} is described in appendix A of the ATP \cite{atp}.

\section{Test case pass/fail criteria}
The acceptance tests as a whole succeed if all the individual acceptance tests pass. Similarly if one test fails, the software is rejected.
The test criteria are described in chapter \ref{chap:testCaseSpecs}.